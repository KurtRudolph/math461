\documentclass{article}
\usepackage{hw_style}
\usepackage{enumerate}
\usepackage{graphicx}
\usepackage{verbatim}

% Homework Specific Information
\newcommand{\hmwkTitle}{Homework \#7}
\newcommand{\hmwkDueDate}{ Monday June 9th, 2012, 9:00 AM}
\newcommand{\hmwkAuthorName}{Kurt Rudolph}%Name:
\newcommand{\hmwkNetID}{rudolph9}%your netid
\newcommand{\hmwkNotes}{}%I worked with...

\newcommand{\hmwkSubTitle}{}
\newcommand{\hmwkClass}{Math 461}
\newcommand{\hmwkClassTime}{}
\newcommand{\hmwkClassInstructor}{Kenneth B. Stolarsky}

\begin{document}
\begin{spacing}{1.1}
\maketitle

%=============================p.224 #1==========================% 
\newpage
\begin{homeworkProblem}
  {\bf Chapter 5, Problem 1}\\
  Let $X$ be a random variable with probability density function 
  \[f(x) =
      \left\{
        \begin{array}{lr}
          c(1 - x^2) & -1 < x < 1 \\ 
          0 & \text{otherwise}
        \end{array}
      \right.\]
  \begin{enumerate}[(a)]
    \item  What is the value of $c$?
      \begin{homeworkSection}{Solution}
        \[\int\limits_{-1}^{1}{ c (1 - x^2) dx} = c \int\limits_{-1}^{1}{ (1 - x^2) dx}\]
        \[= c \left(2 - \frac{ 2}{ 3}\right) = \frac{ 4 c}{ 3}\]
        Since the integral must evaluate to one we find $c = \frac{ 3}{ 4}$.
      \end{homeworkSection}
    \item What is the cumulative distribution function of $X$?
      \begin{homeworkSection}{Solution}
        \[F( x) = \int\limits_{-1}^{ x} \frac{ 3}{ 4} (1 - \xi^2) d \xi
          = \frac{ 3}{ 4} \left(x - \frac{ x^3}{ 3}\right) + \frac{ 1}{ 3}, -1 \le x \le 1\]
      \end{homeworkSection}
  \end{enumerate}
\end{homeworkProblem}

%=============================p.224 #2==========================% 
\newpage
\begin{homeworkProblem}
  {\bf Chapter 5, Problem 2}\\
  A system consisting of one original unit plus a spare can 
  function for a random amount of time $X$. If the density of $X$ 
  is given (in units of months) by
  \[f(x) =
      \left\{
        \begin{array}{lr}
          C x e^{ -x/2} & x > 1 \\ 
          0 & x \le 0
        \end{array}
      \right.\]
  what is the probability that the system functions 
  for at least 5 months?
  \begin{homeworkSection}{Solution}
    Since the distribution function is a \emph{probability density} we have
      \[\int\limits_{0}^{\infty} c x e^{-1/2} dx = 1\]
    Evaluating the integral we find
      \[\int\limits_{0}^{\infty} c x e^{-1/2} dx 
      =  c \left[ \left.x \frac{ e^{-x/2}}{ (-1/2)} \right|_{0}^{\infty} - \frac{ 1}{ (-1/2)} \int_{0}^{\infty} e^{-x/2} dx\right]
      \\= \left. 2 c \frac{ e^{-x/2}}{ (-1/2)} \right|_{0}^{\infty} = 4 c\]
    Hence $c = \frac{ 1}{ 4}$.  Solving for the probability that the system will
    last at least five months we find
      \[\int\limits_{5}^{\infty} \frac{ 1}{ 4} x e^{-x/2} dx = \frac{ 7}{ 2} e^{-5/2}\]
  \end{homeworkSection}
\end{homeworkProblem}

%=============================p.224 #7==========================% 
\newpage
\begin{homeworkProblem}
  {\bf Chapter 5, Problem 7}\\
  The density function of X is given by
  \[f(x) =
      \left\{
        \begin{array}{lr}
          a + b x^2 & 0 \le x \le 1 \\ 
          0 & \text{otherwise}
        \end{array}
      \right.\]
  If $E[ X] = \frac{ 3}{ 5}$, find $a$ and $b$.
  \begin{homeworkSection}{Solution}
    \[\int\limits_{0}^{1} xa + bx^3 dx = \left. \frac{ a x^2}{ 2} \frac{ bx^4}{ 4}\right|_0^1\]
    \[= \frac{ a }{ 2} + \frac{ b}{ 4} = \frac{ 2 a + b}{ 4} = \frac{ 10 a + 5 b}{ 20} \]
    hence 
    \[a = b = 1\]
  \end{homeworkSection}
\end{homeworkProblem}

%=============================p.224 #8==========================% 
\newpage
\begin{homeworkProblem}
  {\bf Chapter 5, Problem 8}\\
   The lifetime in hours of an electronic tube is a random variable 
    having a probability density function given by
  
  \[f(x) = x e^{-x}, x \ge 0\]

  Compute the expected lifetime of such a tube.
  \begin{homeworkSection}{Solution}
    \[E[X] = \int\limits_0^{\infty} x^2 e^{-x} dx = -e^{-x} ( x^2 + 2x + 2) = 2\]

  \end{homeworkSection}
\end{homeworkProblem}

%=============================p.224 #11==========================% 
\newpage
\begin{homeworkProblem}
  {\bf Chapter 5, Problem 11}\\
  A point is chosen at random on a line segment of length $L$. Interpret 
  this statement, and find the probability that the ratio of the shorter 
  to the longer segment is less than $\frac{ 1}{ 4}$
  \begin{homeworkSection}{Solution}
    Interpreting the problem, it states that given randomly picked point $X$
    on a uniformly distributed interval $[0,L]$ find
      \[P\left( \frac{ min( L - x, x)}{ max(L - x, x)} < \frac{ 1}{ 4}\right)\]
    Where $p(x) = \frac{ 1}{ L}$ is our probability density function.

    This probability is true across two regions of $L$, $[0, l_1]$, and $[l_2, L]$
    where $ l_1 = L - l_2$.  Solving for the bound on the first region we find
      \[\int\limits_{0}^{l_1} p(x) dx + \int\limits_{l_2}^{L} p(x) dx 
      = 2 \int\limits_{0}^{l_1} p(x) dx\]
    Since $min(L - x, x) < \frac{ 1}{ 4} max(L - x, x)$ in the desired region
    $[0,l_1]$ and $[l_2, L]$ we find that 
      \[x = \frac{ 1}{ 4} ( L - x)\]
    we find that $\frac{ L}{ 5}$ is the upper bound for the region $[0, L_1]$.  Following
    our previous assertion that the two regions are equal in size we are able to solve
    for the probability 
      \[ 2 \int\limits_{0}^{\frac{ L}{ 5}} \frac{ 1}{ L} dx = \frac{ 2}{ 5}\]

  \end{homeworkSection}
\end{homeworkProblem}
  
%=============================p.224 #18==========================% 
\newpage
\begin{homeworkProblem}
  {\bf Chapter 5, Problem 18}\\
  Suppose that $X$ is a normal random variable with mean 5. If $P\{X > 9\} = 0.2$, 
  approximately what is $Var( X)$?

  \begin{homeworkSection}{Solution}
  Since $X$ is a random variable it follows that 
    \[ P\{ X > 9\} = P\{ \frac{ X - 5}{ \sigma}\} > \frac{ 9 - 5}{ \sigma}\}
    = P\{ Z > \frac{ 4}{ \sigma}\} = 1 - P\{ Z < \frac{ 4}{ \sigma}\}
    = 1 - \Phi\left( \frac{ 4}{ \sigma}\right) = 0.2,\]
    hence
    \[\Phi\left( \frac{ 4}{ \sigma} \right) = 0.8\]
    and
    \[\frac{ 4}{ \sigma} = \Phi^{-1}(0.8) = 0.8416\]
    therefore
    \[\sigma = 4.7527\]
    \[V( X) = \sigma^2 \approx 22.58\]

  \end{homeworkSection}
\end{homeworkProblem}

%=============================p.227 #7==========================% 
\newpage
\begin{homeworkProblem}
  {\bf Chapter 5, Theoretical Exercise 7}\\
  The standard deviation of X, denoted SD(X), is
  given by
      \[SD( X) = \sqrt{ Var(X)}\]
  Find $SD( aX + b)$ if $X$ has variance $\sigma^2$.

  \begin{homeworkSection}{Solution}
    \[SD( aX + b) = \sqrt{ Var( a X + b)} = \sqrt{ Var( a X)} = \sqrt{ a^2 Var( X)} = |a| \sigma\]

  \end{homeworkSection}
\end{homeworkProblem}
  
%=============================p.227 #8==========================% 
\newpage
\begin{homeworkProblem}
  {\bf Chapter 5, Theoretical Exercise 8}\\
  Let $X$ be a random variable that takes on values 
  between $0$ and $c$.  That is, $P\{ 0 \le X \le c\} = 1$.
  \\Show that
    \[Var( X) \le \frac{ c^2}{ 4}\]
  \emph{Hint}: One approach is to first argue that
    \[E[ X^2] \le c E[ X]\]

  \begin{homeworkSection}{Solution}
    Following the hint and since $x \le \forall x \in [0, c]$
    \[E[ X^2] = \int\limits_0^c x^2 f( x) dx \le c \int\limits_0^c x f( x) dx = c E[ X]\]
    \[Var( X) = E[ X^2] - E[ X]^2 \le c E[ X] - E[ X]^2 
    = c ^2 \left(\frac{ E[ X]}{ c} - \frac{ E[ X]^2}{ c^2}\right)\]
    Now we have
    \[Var(X) \le c^2 \left(\frac{ E[X]}{ c} - \left(\frac{ E[X]}{ c}\right)^2\right) 
    = c^2 \frac{ E[X]}{ c} \left(1 - \frac{ E[X]}{ c}\right)\]
    Now we need to select that value for $\frac{ E[X]}{ c}$ which maximizes the previous
    expression.  Taking the derivative and setting it to zero we find
      \[c^2 \left(1 - 2 \frac{ E[X]}{ c}\right) = 0\]
    Hence, $\frac{ E[X]}{ c} = \frac{ 1}{ 2}$ since the second ordered derivative is
    negative it follows that this is a maximum, value at which is $\frac{ 1}{ 4}$.
    \\Therefore
      \[Var( X) \le \frac{ c^2}{ 4}\]

  \end{homeworkSection}
\end{homeworkProblem}
\end{spacing}
\end{document}

\begin{comment}%==========================================================
%=============================Problemi==========================% 
\begin{homeworkProblem}
  
  \begin{homeworkSection}{Solution}
    
  \end{homeworkSection}
\end{homeworkProblem}
%=============================Problemi==========================% 
\begin{homeworkProblem}
  
  \begin{enumerate}[(a)]
    \item 
      \begin{homeworkSection}{Solution}
    
      \end{homeworkSection}
  \end{enumerate}
\end{homeworkProblem}
p.224
1
2
7
8
11
18

p.227
5.7
5.8


\end{comment}

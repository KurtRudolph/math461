\documentclass{article}
\usepackage{hw_style}
\usepackage{enumerate}
\usepackage{graphicx}
\usepackage{verbatim}
\usepackage{amsmath}
\usepackage{mathtools}

% Homework Specific Information
\newcommand{\hmwkTitle}{Homework \#11}
\newcommand{\hmwkDueDate}{Friday, July 20, 2012}
\newcommand{\hmwkAuthorName}{Kurt Rudolph}%Name:
\newcommand{\hmwkNetID}{rudolph9}%your netid
\newcommand{\hmwkNotes}{}%I worked with...

\newcommand{\hmwkSubTitle}{}
\newcommand{\hmwkClass}{Math 461}
\newcommand{\hmwkClassTime}{}
\newcommand{\hmwkClassInstructor}{Kenneth B. Stolarsky}

\begin{document}
\begin{spacing}{1.1}
\maketitle
%=============================p.375 #36==========================% 
\newpage
\begin{homeworkProblem}
  {\bf Chapter 7, Exercise 36}\\
  Let $X$ be the number of 1's and Y the number of 2's that occur in 
  $n$ rolls of a fair die. Compute $Cov( X, Y)$.
  \begin{homeworkSection}{Solution}
    
  \end{homeworkSection}
\end{homeworkProblem}

%=============================p.375 #37==========================% 
\newpage
\begin{homeworkProblem}
  {\bf Chapter 7, Exercise 37}\\
  A die is rolled twice. Let $X$ equal the sum of the outcomes, and 
  let $Y$ equal the first outcome minus the second. Compute $Cov(X, Y)$.
  \begin{homeworkSection}{Solution}
    
  \end{homeworkSection}
\end{homeworkProblem}

%=============================p.376 #40==========================% 
\newpage
\begin{homeworkProblem}
  {\bf Chapter 7, Exercise 40}\\
  The joint density function of $X$ and $Y$ is given by
    \[f( x, y) = \frac{ 1}{ y} e^{-(y + x / y)}, x > 0, y > 0\]
  Find $E[ X], E[ Y]$, and show $Cov( X, Y) = 1$.
  \begin{homeworkSection}{Solution}
    
  \end{homeworkSection}
\end{homeworkProblem}

%=============================p.380 #4==========================% 
\newpage
\begin{homeworkProblem}
  {\bf Chapter 7, Theoretical Exercise 4}\\
  Let $X$ be a random variable having finite expectation $\mu$ and 
  variance $\sigma^2$, and let $g(\cdot)$ be a twice differentiable function. 
  Show that
  \[E[ g( X)] \approx g( \mu) + \frac{ g''(\mu)}{ 2} \sigma^2\]
  \emph{Hint}: Expand $g(\cdot)$ in a Taylor series about $\mu$. Use the 
  first three terms and ignore the remainder.
  \begin{homeworkSection}{Solution}
    
  \end{homeworkSection}
\end{homeworkProblem}
  
\end{spacing}
\end{document}

\begin{comment}%==========================================================
%=============================Problemi==========================% 
\newpage
\begin{homeworkProblem}
  
  \begin{homeworkSection}{Solution}
    
  \end{homeworkSection}
\end{homeworkProblem}
%=============================Problemi==========================% 
\newpage
\begin{homeworkProblem}
  
  \begin{enumerate}[(a)]
    \item 
      \begin{homeworkSection}{Solution}
    
      \end{homeworkSection}
  \end{enumerate}
\end{homeworkProblem}

\documentclass{article}
\usepackage{hw_style}
\usepackage{enumerate}
\usepackage{graphicx}
\usepackage{verbatim}
\usepackage{amsmath}

% Homework Specific Information
\newcommand{\hmwkTitle}{Homework \#9}
\newcommand{\hmwkDueDate}{ Monday, July 16, 2012, 9:00AM}
\newcommand{\hmwkAuthorName}{Kurt Rudolph}%Name:
\newcommand{\hmwkNetID}{rudolph9}%your netid
\newcommand{\hmwkNotes}{}%I worked with...

\newcommand{\hmwkSubTitle}{}
\newcommand{\hmwkClass}{Math 461}
\newcommand{\hmwkClassTime}{}
\newcommand{\hmwkClassInstructor}{Kenneth B. Stolarsky}

\begin{document}
\begin{spacing}{1.1}
\maketitle

%=============================p.228 #18==========================% 
\newpage
\begin{homeworkProblem}
  {\bf Chapter 5, Theoretical Exercise 18}\\
  Verify that the gamma density function integrates to 1.
  
  \begin{homeworkSection}{Definitions}
    
    
  \end{homeworkSection}
  \begin{homeworkSection}{Solution}
    
  \end{homeworkSection}
\end{homeworkProblem}

%=============================p.228 #19==========================% 
\newpage
\begin{homeworkProblem}
  {\bf Chapter 5, Theoretical Exercise 19}\\
  If $X$ is an exponential random variable with mean $1/\lambda$, show that
  \[E[ X^k] = \frac{ k!}{ \lambda^k}, k = 1, 2, \dots\]
  \emph{Hint}: Make use of the gamma density function to evaluate the preceding.
  \begin{homeworkSection}{Definitions}
    
    
  \end{homeworkSection}
  \begin{homeworkSection}{Solution}
    
  \end{homeworkSection}
\end{homeworkProblem}

%=============================p.228 #20==========================% 
\newpage
\begin{homeworkProblem}
  {\bf Chapter 5, Theoretical Exercise 20}\\
  Verify that
  \[Var( X) = \frac{ \alpha}{ \lambda^2}\]
  when $X$ is a gamma random variable with parameters $\alpha$ and $\lambda$.
  \begin{homeworkSection}{Definitions}
    
    
  \end{homeworkSection}
  \begin{homeworkSection}{Solution}
    
  \end{homeworkSection}
\end{homeworkProblem}

%=============================p.228 #21==========================% 
\newpage
\begin{homeworkProblem}
  {\bf Chapter 5, Theoretical Exercise 21}\\
  Show that 
  \[\Gamma \left(\frac{ 1}{ 2}\right) = \sqrt{ \pi}\]
  \emph{Hint}: $\Gamma\left( \frac{ 1}{ 2}\right) = \int\limits_0^{\infty} e^{-x} x^{-1/2} dx$. 
  Make the change of variables $y = sqrt{ 2x}$ and then relate the resulting expression
  to the normal distribution.
  \begin{homeworkSection}{Definitions}
    
    
  \end{homeworkSection}
  \begin{homeworkSection}{Solution}
    
  \end{homeworkSection}
\end{homeworkProblem}

%=============================p.228 #26==========================% 
\newpage
\begin{homeworkProblem}
  {\bf Chapter 5, Theoretical Exercise 26}\\
  If $X$ is a beta random variable with parameters $a$ and $b$, show that
  \begin{align*}
    E[ X]   &= \frac{ a}{ a + b}\\
    Var( X) &= \frac{ a b}{ (a + b)^2 (a + b + 1)}
  \end{align*}
  Use p.218(6.3) and $\Gamma( n + 1) = n \Gamma( n)$
  \begin{homeworkSection}{Definitions}
    
    
  \end{homeworkSection}
  \begin{homeworkSection}{Solution}
    
  \end{homeworkSection}
\end{homeworkProblem}

%=============================p.287 #9==========================% 
\newpage
\begin{homeworkProblem}
  {\bf Chapter 6, Exercise 9}\\
  The joint probability density function of $X$ and $Y$ is given by
  \[f( x, y) = \frac{ 6}{ 7} \left( x^2 + \frac{ x y}{ 2}\right), 
    0 < x < 1, 0 < y < 2\]

  \begin{homeworkSection}{Definitions}

  \end{homeworkSection}
  \begin{enumerate}[(a)]
    \item Verify that this is indeed a joint density function.
      \begin{homeworkSection}{Solution}

      \end{homeworkSection}
    \item Compute the density function of $X$.
      \begin{homeworkSection}{Solution}

      \end{homeworkSection}
    \item Find $P\{ X > Y\}$.
      \begin{homeworkSection}{Solution}

      \end{homeworkSection}
    \item Find $P\{ Y > \frac{ 1}{ 2}| X < \frac{ 1}{ 2}\}$.
      \begin{homeworkSection}{Solution}

      \end{homeworkSection}
    \item Find $E[ X]$
      \begin{homeworkSection}{Solution}

      \end{homeworkSection}
    \item Find $E[ Y]$
      \begin{homeworkSection}{Solution}

      \end{homeworkSection}
  \end{enumerate}
\end{homeworkProblem}

%=============================p.287 #10==========================% 
\newpage
\begin{homeworkProblem}
  {\bf Chapter 6, Exercise 10}\\
  The joint probability density function of $X$ and $Y$ is given by
    \[f( x, y) = e^{-(x + y)}, 0 \le x < \infty, 0 \le y < \infty\]
  Find 
  \begin{enumerate}[(a)]
    \item $P\{ X < Y\}$
      \begin{homeworkSection}{Solution}
        
      \end{homeworkSection}
    \item $P\{ X < a\}$
      \begin{homeworkSection}{Solution}
        
      \end{homeworkSection}
  \end{enumerate}
\end{homeworkProblem}

%=============================p.288 #22==========================% 
\newpage
\begin{homeworkProblem}
  {\bf Chapter 6, Exercise 23}\\
  The random variables $X$ and $Y$ have joint density function
  \[f( x, y) = 12 x y (1 - x), 0 < x < 1, 0 < y < 1\]
  and equal to 0 otherwise.
  \begin{enumerate}[(a)]
    \item Are $X$ and $Y$ independent?
      \begin{homeworkSection}{Solution}
        
      \end{homeworkSection}
    \item Find $E[ X]$
      \begin{homeworkSection}{Solution}
        
      \end{homeworkSection}
    \item Find $E[ Y]$
      \begin{homeworkSection}{Solution}
        
      \end{homeworkSection}
    \item Find $Var( X)$
      \begin{homeworkSection}{Solution}
        
      \end{homeworkSection}
    \item Find $Var( Y)$
      \begin{homeworkSection}{Solution}
        
      \end{homeworkSection}
  \end{enumerate}
\end{homeworkProblem}

%=============================p.289 #30==========================% 
\newpage
\begin{homeworkProblem}
  {\bf Chapter 6, Exercise 30}\\
  Jill's bowling scores are approximately normally distributed with 
  mean 170 and standard deviation 20, while Jack's scores are 
  approximately normally distributed with mean 160 and standard 
  deviation 15. If Jack and Jill each bowl one game, then assuming 
  that their scores are independent random variables, approximate 
  the probability that
  \begin{enumerate}[(a)]
    \item Jack's score is higher
      \begin{homeworkSection}{Solution}
        
      \end{homeworkSection}
    \item The total of their scores is above 350
      \begin{homeworkSection}{Solution}
        
      \end{homeworkSection}
  \end{enumerate}
\end{homeworkProblem}

\end{spacing}
\end{document}

\begin{comment}%==========================================================
* p.228
  * 18
  * 19
  * 20
  * 21
  * 26
      * Use p.218(6.3)  and \[\Gamma( n + 1) = n \Gamma( n)\]
* p.287
  * 9
  * 10
  * 23
  * 28
      * exp. significant
  * 30 
      * exp. significant
%=============================Problemi==========================% 
\newpage
\begin{homeworkProblem}
  
  \begin{homeworkSection}{Solution}
    
  \end{homeworkSection}
\end{homeworkProblem}
%=============================Problemi==========================% 
\newpage
\begin{homeworkProblem}
  
  \begin{enumerate}[(a)]
    \item 
      \begin{homeworkSection}{Solution}
    
      \end{homeworkSection}
  \end{enumerate}
\end{homeworkProblem}

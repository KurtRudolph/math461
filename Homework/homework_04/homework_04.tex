\documentclass{article}
\usepackage{hw_style}
\usepackage{enumerate}
\usepackage{graphicx}
\usepackage{verbatim}

% Homework Specific Information
\newcommand{\hmwkTitle}{Homework \#4}
\newcommand{\hmwkDueDate}{Friday, June 22 9:00AM }
\newcommand{\hmwkAuthorName}{Kurt Rudolph}%Name:
\newcommand{\hmwkNetID}{rudolph9}%your netid
\newcommand{\hmwkNotes}{}%I worked with...

\newcommand{\hmwkSubTitle}{}
\newcommand{\hmwkClass}{Math 461}
\newcommand{\hmwkClassTime}{}
\newcommand{\hmwkClassInstructor}{Kenneth B. Stolarsky}

\begin{document}
\begin{spacing}{1.1}
\maketitle

%=============================p.012 problem18=========================% 
\newpage
\begin{homeworkProblem}
  {\bf Chapter 3, Probem 18}\\
  A total of 46 percent of the voters in a certain city classify 
  themselves as Independents, whereas 30 percent classify themselves 
  as Liberals and 24 percent say that they are Conservatives. In a 
  recent local election, 35 percent of the Independents, 62 percent 
  of the Liberals, and 58 percent of the Conservatives voted. A voter 
  is chosen at random. 

  Let
  \begin{itemize}
    \item $I$ be the group of Independent voters.
    \item $L$ be the group of Liberal voters.
    \item $R$ be the group of Republican voters.
    \item $V$ be the group of voters who actually voted.
  \end{itemize}
  Given
  \begin{itemize}
    \item $P( I) = .46$ 
    \item $P( L) = .30$ 
    \item $P( R) = .24$
    \item $P( V|I) = .35$
    \item $P( V|L) = .62$
    \item $P( V|R) = .58$
  \end{itemize}
  Then let
    \[P( V) = P( I) P( V|I) + P( L) P( V|L) + P( R) P( V|R) = .4862\]

  Given that this person voted in the local election, what is the 
  probability that he or she is
  \begin{enumerate}[(a)]
    \item an Independent?
      \begin{homeworkSection}{Solution}
         \[P( I|V) = \frac{ P( I) P( V|I)}{ P(V)} = .331139\]
      \end{homeworkSection}
    \item a Liberal?
      \begin{homeworkSection}{Solution}
         \[P( L|V) = \frac{ P( L)( V|L)}{ P(V)} = .382559\]
      \end{homeworkSection}
    \item a Conservative?
      \begin{homeworkSection}{Solution}
         \[P( R|V) = \frac{ P( R)( V|R)}{ P(V)} = .286302\]
      \end{homeworkSection}
    \item What fraction of voters participated in thelocal election?
      \begin{homeworkSection}{Solution}
         \[P( V) = .4862\]
      \end{homeworkSection}
  \end{enumerate}
\end{homeworkProblem}
%=============================p.105 42==========================% 
\newpage
\begin{homeworkProblem}
 {\bf Chapter 3, Probem 42}\\
  Three cooks, $A$, $B$, and $C$, bake a special kind of cake, and with 
  respective probabilities $.02$, $.03$, and $.05$, it fails to rise. 
  In the restaurant where they work, $A$ bakes $50$ percent of these cakes, 
  $B$ $30$ percent, and $C$ $20$ percent. What proportion of ''failures'' 
  is caused by $A$?
  \begin{homeworkSection}{Solution}
     Let
    \begin{itemize}
      \item $A$ be the group of cakes the first cook bakes
      \item $B$ be the group of cakes the second cook bakes
      \item $C$ be the group of cakes the third cook bakes
      \item $F$ be the group of cakes which fail
    \end{itemize}
    Given
    \begin{itemize}
      \item $P( A) = .5$
      \item $P( B) = .3$
      \item $P( C) = .2$
      \item $P( F|A) = .02$
      \item $P( F|B) = .03$
      \item $P( F|C) = .05$
    \end{itemize}
    Then
      \[P( F) = P( A)P( F|A) + P( B)P( F|B) + P( C)P( F|C) = .029\]
      \[P( A|F)= \frac{ P( A)P( F|A)}{ P( F)} = 0.344\]
  \end{homeworkSection}
\end{homeworkProblem}
%=============================p.105 43==========================% 
\newpage
\begin{homeworkProblem}
  {\bf Chapter3, Problem 43}\\
  There are 3 coins in a box. One is a two-headed coin, another is a fair coin, 
  and the third is a biased coin that comes up heads 75 percent of the time. 
  When one of the 3 coins is selected at random and flipped, it shows heads. 
  What is the probability that it was the two-headed coin? 

  \begin{homeworkSection}{Solution}
     Let
    \begin{itemize}
      \item $C_1$ be the two-heded coin
      \item $C_2$ be the fair coin
      \item $C_3$ be the biased coin
      \item $H$ be the event heads occurs
      \item $T \equiv H^c$ be the event tails occurs
    \end{itemize}
    Given
    \begin{itemize}
      \item $P( H|C_1) = 1.0$
      \item $P( H|C_2) = .75$
      \item $P( H|C_3) = .5$
    \end{itemize}

    \[P( C_1|H) = \frac{ P( C_1) P( H|C_1)}{ P( H)} 
    = \frac{ P( C_1) P( H|C_1)}{ P( C_1) P( H|C_1) + P( C_2) P( H|C_2) + P( C_3) P( H|C_3)} 
    = \frac{ \frac{ 1}{ 3}}{ \frac{ 1}{ 3} + \frac{ 1}{ 3} \frac{ 3}{ 4} + \frac{ 1}{ 3} \frac{ 1}{ 2}}\]
  \end{homeworkSection}
\end{homeworkProblem}
%=============================p.107 67==========================% 
\newpage
\begin{homeworkProblem}
  {\bf Chapter 3, Problem 67}\\
  An engineering system consisting of $n$ components is said to be a $k$-out-of-$n$ system $(k \le n)$ if 
  the system functions if and only if at least $k$ of the $n$ components function. Suppose that all 
  components function \emph{independently} of each other.
  \begin{enumerate}[(a)]
    \item If the $i$th component functions with probability $P_i,i = 1, 2, 3, 4$, compute the 
          probability that a $2$-out-of-$4$ system functions. 
      \begin{homeworkSection}{Solution}
        Given
          \begin{itemize}
            \item $k = 2$ is the least number of components functioning.
            \item $n = 4$ is the total number of components in the system.
            \item $P_i$ is the probability that the $i$th component funtions where $ 1 \le i \le 4$
          \end{itemize}
        Let
          \begin{itemize}
            \item $F$ be the event the system functions
          \end{itemize}
        Then
          \[P( F|\{k,n\}) = P( F|\{2,4\}) = \sum\limits_{i = 1}^{n - 1}{ \sum\limits_{j = i}^{n}{ P_i P_j}} = \sum\limits_{i = 1}^{3}{ \sum\limits_{j = i}^{4}{ P_i P_j}}\]
    
      \end{homeworkSection}
    \item[(c)] Repeat for a k-out-of-n system when all the Pi equal p (that is, Pi = p,i = 1,2,...,n).
      \begin{homeworkSection}{Solution}
        Given
          \begin{itemize}
            \item $k$ is the least number of components functioning.
            \item $n$ is the total number of components in the system.
            \item $P_i$ is the probability that the $i$th component funtions where $ 1 \le i \le n$
          \end{itemize}
        Let
          \begin{itemize}
            \item $F$ be the event the system functions
            \item $0 \le i \le n - 1$
            \item $p$
            \item $k = k - 1$
          \end{itemize}
        Then
          \[P( F|\{k,n\}) = \sum\limits_{i_0 = 1}^{n - k + 1}{ \sum\limits_{i_1 = i_0}^{n - k + 2}{ \cdots \sum\limits_{i_{k - 2} = i_{k - 3}}^{n - k + i_{k - 3}}{ \sum\limits_{i_{k - 1} = j_{k - 2}}^{n}{ \left(\prod\limits_{l = i_0}^{i + k - 1}{ p_l}\right)}}}}\]
      \end{homeworkSection}
  \end{enumerate}
      \begin{homeworkSection}{Comments}
        Why is this problem phrased ``if and only if at least''?  The if and only if seems unnecicary if \emph{at least} is included.
      \end{homeworkSection}
\end{homeworkProblem}
%=============================p.109 76==========================% 
\newpage
\begin{homeworkProblem}
  {\bf Chapter 3, Problem 76}\\
  Suppose that $E$ and $F$ are mutually exclusive events of an experiment. 
  Show that if independent trials of this experiment are performed, 
  then $E$ will occur before $F$ with probability $\frac{ P( E)}{P( E) + P( F)}.$
  \begin{homeworkSection}{Solution}
    Since
      \[P( E^c) = P( F)\]
    Therefore
      \[\frac{ P( E)}{ P( E) + P( F)} = \frac{ P( E)}{ 1 - P( F) + P( F)} = \frac{ P( E)}{ 1} = P( E)\]
    
  \end{homeworkSection}
\end{homeworkProblem}
\begin{comment}
If $E$ and $F$ are mutually exclusive events in an experiment, 
then $P( E \cup F) = P( E) + P( F)$. We desire to compute the probability that $E$ occurs
before $F$ , which we will denote by $p$. To compute $p$ we condition on the
three mutually exclusive events $E$, $F$ , or $(E \cup F )^c$. This last event
are all the outcomes not in $E$ or $F$. Letting the event $A$ be the event
that $E$ occurs before $F$, we have that
\[p = P( A|E) P( E) + P( A|F) P(F ) + P( A|(E \cup F )^c) P( (E \cup F )^c)\] . 

\[P( A|E) = 1\]
\[P( A|F) = 0\] 
\[P( A|(E \cup F)^c) = p\], 

since if neither E or F happen the next
experiment will have $E$ before $F$ (and thus event $A$ with probability $p$).
Thus we have that 

\[p = P( E) + p P( (E \cup F)^c)\]
\[= P( E) + p (1 − P( E \cup F))\]
\[= P( E) + p (1 − P( E) − P( F))\]

Solving for $p$ gives 
\[p = \frac{ P( E)}{ P( E) + P( F)\]
as we were to show. 

\end{comment}
%=============================p.111 4==========================% 
\newpage
\begin{homeworkProblem}
  {\bf Chapter 3, Theoretical Exercise 4}\\
  A ball is in anyone of $n$ boxes and is in the $i$th box with 
  probability $P_i$. If the ball is in box $i$, a search of that 
  box will uncover it with probability $\alpha_i$. Show that the 
  conditional probability that the ball is in box $j$, given that 
  a search of box $i$ did not uncover it, is \\
  If $j \neq i$
    \[\frac{ P_j}{ 1 - \alpha_i P_i}\]
  If $j = i$
    \[\frac{ (1 - \alpha_i) P_i}{ 1 - \alpha_i P_i}\]
  \begin{homeworkSection}{Solution}
    
  \end{homeworkSection}
\end{homeworkProblem}
\end{spacing}
\end{document}





\begin{comment}%==========================================================
%=============================Problemi==========================% 
\newpage
\begin{homeworkProblem}
  
  \begin{homeworkSection}{Solution}
    
  \end{homeworkSection}
\end{homeworkProblem}
%=============================Problemi==========================% 
\begin{homeworkProblem}
  
  \begin{enumerate}[(a)]
    \item 
      \begin{homeworkSection}{Solution}
    
      \end{homeworkSection}
  \end{enumerate}
\end{homeworkProblem}
\end{comment}

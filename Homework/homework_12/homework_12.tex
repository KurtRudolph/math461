\documentclass{article}
\usepackage{hw_style}
\usepackage{enumerate}
\usepackage{graphicx}
\usepackage{verbatim}
\usepackage{amsmath}
\usepackage{mathtools}

% Homework Specific Information
\newcommand{\hmwkTitle}{Homework \#12}
\newcommand{\hmwkDueDate}{Tuesday, July 24, 2012}
\newcommand{\hmwkAuthorName}{Kurt Rudolph}%Name:
\newcommand{\hmwkNetID}{rudolph9}%your netid
\newcommand{\hmwkNotes}{}%I worked with...

\newcommand{\hmwkSubTitle}{}
\newcommand{\hmwkClass}{Math 461}
\newcommand{\hmwkClassTime}{}
\newcommand{\hmwkClassInstructor}{Kenneth B. Stolarsky}

\begin{document}
\begin{spacing}{1.1}
\maketitle
%=============================Problem 1==========================% 
\newpage
\begin{homeworkProblem}
  Calculate the $M(t)$ for the uniform distribution (i.e. verfy this 
  item of the p.359 table) and use it to find the mean, variance 
  and 3rd moment.
  \begin{homeworkSection}{Definitions}
    {\bf Moment Generating Function}\\
      The moment generating function $M(t)$ of the random variable $X$ 
      is defined for all real values of $t$ by
      \[
        M(t) = E[ e^{tx}] = \left\{ \begin{array}{l l}
          \sum\limits_x e^{tx} p(x) &\text{if $X$ is discrete with mass function $p( x)$}\\
          \int\limits_{-\infty}^\infty e^{tx} f( x) dx 
              &\text{if $X$ is continuous with density $f( x)$}
        \end{array}\right.
      \]
    {\bf Uniform Distrobution}\\
    Uniform over $(a,b)$
      \[
        f( x) = \left\{ \begin{array}{l l}
          \frac{ 1}{ b - a} & a < x < b\\
          0 & otherwise
        \end{array} \right.
      \]
    \[M(t) = \frac{ e^{tb} - e^{ta}}{ t(b - a)}\]
  \end{homeworkSection}
  \begin{homeworkSection}{Solution}
    \begin{align*}
      M( t) &= \int\limits_a^b e^{t x} f(x) dx\\
      &= \int\limits_a^b e^{t x} \frac{ 1}{ b - a} dx\\
      &= \left. e^{t x} \frac{ 1}{ t(b - a)} \right|_a^b\\
      &= \frac{ e^{t b} - e^{t a}1}{ t(b - a)}\\
    \end{align*}
    \begin{align*}
      \[M'(0) = E[X] = t\frac{ e^{t b} - e^{t a}1}{ t(b - a)}\]
      \[M'''(t) = t^3\frac{ e^{t b} - e^{t a}1}{ t(b - a)}\]
    \end{align*}
  \end{homeworkSection}
\end{homeworkProblem}

%=============================Problem 2==========================% 
\newpage
\begin{homeworkProblem}
  Use the appropriate $M( t)$ to calulate the 3rd power moments

  \begin{enumerate}[(1)]
    \item A binomial $(n, p)$ random variable
      \begin{homeworkSection}{Solution}
        
      \end{homeworkSection}
    \item An exponential $(\lambda)$ random variable
      \begin{homeworkSection}{Solution}
        
      \end{homeworkSection}
  \end{enumerate}
\end{homeworkProblem}

%=============================Problem3==========================% 
\newpage
\begin{homeworkProblem}
  Simplify as much as possible $\left(\frac{ m_4}{ \sigma^4}\right) - 3$ 
  for an $n(\mu, \sigma^2)$ random variable, where here the $m_4$ is the
  4th power moment.
  \begin{homeworkSection}{Solution}
    
  \end{homeworkSection}
\end{homeworkProblem}

%=============================Problem4==========================% 
\newpage
\begin{homeworkProblem}
  Use the \emph{binomial expansion} to derive the $M(t)$ for a negagtive
  binomial $(r, p)$ random variable.
  \begin{homeworkSection}{Solution}
    
  \end{homeworkSection}
\end{homeworkProblem}
%=============================Problem5==========================% 
\newpage
\begin{homeworkProblem}
  Use 4 to verify the formulas for mean and variance given on p.385 
  (last row of the table).
  \begin{homeworkSection}{Solution}
    
  \end{homeworkSection}
\end{homeworkProblem}
  
\end{spacing}
\end{document}

\begin{comment}%==========================================================
%=============================Problemi==========================% 
\newpage
\begin{homeworkProblem}
  
  \begin{homeworkSection}{Solution}
    
  \end{homeworkSection}
\end{homeworkProblem}
%=============================Problemi==========================% 
\newpage
\begin{homeworkProblem}
  
  \begin{enumerate}[(a)]
    \item 
      \begin{homeworkSection}{Solution}
    
      \end{homeworkSection}
  \end{enumerate}
\end{homeworkProblem}

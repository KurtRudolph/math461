\documentclass{article}
\usepackage{hw_style}
\usepackage{enumerate}
\usepackage{graphicx}
\usepackage{verbatim}
\usepackage{amsmath}
\usepackage{mathtools}

% Homework Specific Information
\newcommand{\hmwkTitle}{Homework \#12}
\newcommand{\hmwkDueDate}{Tuesday, July 24, 2012}
\newcommand{\hmwkAuthorName}{Kurt Rudolph}%Name:
\newcommand{\hmwkNetID}{rudolph9}%your netid
\newcommand{\hmwkNotes}{}%I worked with...

\newcommand{\hmwkSubTitle}{}
\newcommand{\hmwkClass}{Math 461}
\newcommand{\hmwkClassTime}{}
\newcommand{\hmwkClassInstructor}{Kenneth B. Stolarsky}

\begin{document}
\begin{spacing}{1.1}
\maketitle
%=============================Problem 1==========================% 
\newpage
\begin{homeworkProblem}
  Calculate the $M(t)$ for the uniform distribution (i.e. verfy this 
  item of the p.359 table) and use it to find the mean, variance 
  and 3rd moment.
  \begin{homeworkSection}{Definitions}
    {\bf Moment Generating Function}\\
      The moment generating function $M(t)$ of the random variable $X$ 
      is defined for all real values of $t$ by
      \[
        M(t) = E[ e^{tx}] = \left\{ \begin{array}{l l}
          \sum\limits_x e^{tx} p(x) &\text{if $X$ is discrete with mass function $p( x)$}\\
          \int\limits_{-\infty}^\infty e^{tx} f( x) dx 
              &\text{if $X$ is continuous with density $f( x)$}
        \end{array}\right.
      \]
    {\bf Uniform Distrobution}\\
    Uniform over $(a,b)$
      \[
        f( x) = \left\{ \begin{array}{l l}
          \frac{ 1}{ b - a} & a < x < b\\
          0 & otherwise
        \end{array} \right.
      \]
    \[M(t) = \frac{ e^{tb} - e^{ta}}{ t(b - a)}\]
  \end{homeworkSection}
  \begin{homeworkSection}{Solution}
    \begin{align*}
      M( t) &= \int\limits_a^b e^{t x} f(x) dx\\
      &= \int\limits_a^b e^{t x} \frac{ 1}{ b - a} dx\\
      &= \left. e^{t x} \frac{ 1}{ t(b - a)} \right|_a^b\\
      &= \frac{ e^{t b} - e^{t a}1}{ t(b - a)}\\
    \end{align*}
    \begin{align*}
      \text{Building off the Series Expansions } &\text{of the first and second derivatives}\\
      M'(t) &= \frac{ a + b}{ 2} + \frac{ t (a^2 + ab + b^2)}{ 3} + k\\
      M''(t) &= \frac{ a^2 + ab + b^2}{ 3} + \frac{ t (a^3 + a^2 b + a b^2 + b^3)}{ 4} + k\\
      E[X] &= M'(0)\\
      &= \frac{ a + b}{ 2}\\
      Var(X) &= E[ X^2] - (E[x])^2\\
      &= M''(0) - (M'(0))^2\\
      &= \frac{ a^2 + ab + b^2}{ 3} - (\frac{ a + b}{ 2})^2\\
      &= \frac{ a^2 + ab + b^2}{ 3} - \frac{ a^2 + 2ab + b^2}{ 4}\\
      &= \frac{ 4a^2 + 4ab + 4b^2}{ 12} - \frac{ 3a^2 + 6ab + 3b^2}{ 12}\\
      &= \frac{ a^2 - 2ab + b^2}{ 12}\\
      &= \frac{ (a - b)^2}{ 12}\\
      &= \frac{ (b - a)^2}{ 12}\\
    \end{align*}
  \end{homeworkSection}
\end{homeworkProblem}

%=============================Problem 2==========================% 
\newpage
\begin{homeworkProblem}
  Use the appropriate $M( t)$ to calulate the 3rd power moments

  \begin{enumerate}[(1)]
    \item A binomial $(n, p)$ random variable
      \begin{homeworkSection}{Definitions}
        {\bf Binomial Random Variable}
          \[p( x) = {n \choose x} p^x (1 - p)^{n - x}\]
      \end{homeworkSection}
      \begin{homeworkSection}{Solution}
       \begin{align*}
          M( t) &= E[ e^{tx}]\\
          &= \sum\limits_{ x = 0}^n {n \choose x} e^{tx} p^x (1 - p)^{n - x}\\
          &= \sum\limits_{ x = 0}^n {n \choose x} (p e^t)^x (1 - p)^{n - x}\\
          &= (p e^t + 1 -p)^n\\
          \text{as shown in the text}
          M'( t) &= n (p e^t + 1 - p)^{n - 1} p e^t\\
          M''( t) &= n (n - 1) (p e^t + 1 - p)^{n - 2} (p e^t)^2 + n (p e^t + 1 - p)^{n - 1} p e^t\\
          \text{Differntiating again}
          M'''( t) 
          & = (n-1) n p^2 (frac{ d}{ dt}(e^{(2 t)} (p e^t - p+1)^{n - 2})) 
            + n p (\frac{ d}{ dt} (e^t (p e^t - p+1)^{n - 1} ))
  | Use the product rule, \frac{ d}{ dt} (u v) = v ( du)/( dt) + u ( dv)/( dt), where u = e^{2 t}  and v = (p e^t - p+1)^{n - 2} :
= | (n - 1) n p^2 ((p e^t - p+1)^{n - 2}  (\frac{ d}{ dt} (e^{2 t} )) + e^{2 t}  (\frac{ d}{ dt} ((p e^t - p+1)^{n - 2} ))) + n p (\frac{ d}{ dt} (e^t (p e^t - p+1)^{n - 1} ))
  | Use the product rule, \frac{ d}{ dt} (u v) = v ( du)/( dt) + u ( dv)/( dt), where u = e^t and v = (p e^t - p+1)^{n - 1} :
= | (n - 1) n p^2 ((p e^t - p+1)^{n - 2}  (\frac{ d}{ dt} (e^{2 t} )) + e^{2 t}  (\frac{ d}{ dt} ((p e^t - p+1)^{n - 2} ))) + n p ((p e^t - p+1)^{n - 1}  (\frac{ d}{ dt} (e^t)) + e^t (\frac{ d}{ dt} ((p e^t - p+1)^{n - 1} )))
  | Use the chain rule, \frac{ d}{ dt} (e^{2 t} ) = ( de^u)/( du) ( du)/( dt), where u = 2 t and ( de^u)/( du) = e^u:
= | (n - 1) n p^2 ((p e^t - p+1)^{n - 2}  (e^{2 t}  (\frac{ d}{ dt} (2 t))) + e^{2 t}  (\frac{ d}{ dt} ((p e^t - p+1)^{n - 2} ))) + n p ((p e^t - p+1)^{n - 1}  (\frac{ d}{ dt} (e^t)) + e^t (\frac{ d}{ dt} ((p e^t - p+1)^{n - 1} )))
  | The derivative of e^t is e^t:
= | (n - 1) n p^2 (e^{2 t}  (p e^t - p+1)^{n - 2}  (\frac{ d}{ dt} (2 t)) + e^{2 t}  (\frac{ d}{ dt} ((p e^t - p+1)^{n - 2} ))) + n p (e^t (\frac{ d}{ dt} ((p e^t - p+1)^{n - 1} )) + e^t (p e^t - p+1)^{n - 1} )
  | Use the chain rule, \frac{ d}{ dt} ((p e^t - p+1)^{n - 2} ) = ( du^{ - 2+n} )/( du) ( du)/( dt), where u = p e^t - p+1 and ( du^{ - 2+n} )/( du) = ( - 2+n) u^{ - 3+n} :
= | (n - 1) n p^2 (e^{2 t}  (p e^t - p+1)^{n - 2}  (\frac{ d}{ dt} (2 t)) + e^{2 t}  ((n - 2) (p e^t - p+1)^{n - 3}  (\frac{ d}{ dt} (p e^t - p+1)))) + n p (e^t (\frac{ d}{ dt} ((p e^t - p+1)^{n - 1} )) + e^t (p e^t - p+1)^{n - 1} )
  | Use the chain rule, \frac{ d}{ dt} ((p e^t - p+1)^{n - 1} ) = ( du^{ - 1+n} )/( du) ( du)/( dt), where u = p e^t - p+1 and ( du^{ - 1+n} )/( du) = ( - 1+n) u^{ - 2+n} :
= | (n - 1) n p^2 ((n - 2) e^{2 t}  (p e^t - p+1)^{n - 3}  (\frac{ d}{ dt} (p e^t - p+1)) + e^{2 t}  (p e^t - p+1)^{n - 2}  (\frac{ d}{ dt} (2 t))) + n p (e^t ((n - 1) (p e^t - p+1)^{n - 2}  (\frac{ d}{ dt} (p e^t - p+1))) + e^t (p e^t - p+1)^{n - 1} )
  | Differentiate the sum term by term and factor out constants:
= | (n - 1) n p^2 ((n - 2) e^{2 t}  (p e^t - p+1)^{n - 3}  (p (\frac{ d}{ dt} (e^t)) - \frac{ d}{ dt} (p) + \frac{ d}{ dt} (1)) + e^{2 t}  (p e^t - p+1)^{n - 2}  (\frac{ d}{ dt} (2 t))) + n p ((n - 1) e^t (p e^t - p+1)^{n - 2}  (\frac{ d}{ dt} (p e^t - p+1)) + e^t (p e^t - p+1)^{n - 1} )
  | Differentiate the sum term by term and factor out constants:
= | (n - 1) n p^2 ((n - 2) e^{2 t}  (p e^t - p+1)^{n - 3}  (p (\frac{ d}{ dt} (e^t)) - \frac{ d}{ dt} (p) + \frac{ d}{ dt} (1)) + e^{2 t}  (p e^t - p+1)^{n - 2}  (\frac{ d}{ dt} (2 t))) + n p ((n - 1) e^t (p e^t - p+1)^{n - 2}  (p (\frac{ d}{ dt} (e^t)) - \frac{ d}{ dt} (p) + \frac{ d}{ dt} (1)) + e^t (p e^t - p+1)^{n - 1} )
  | The derivative of 1 is zero:
= | (n - 1) n p^2 ((n - 2) e^{2 t}  (p e^t - p+1)^{n - 3}  (p (\frac{ d}{ dt} (e^t)) - \frac{ d}{ dt} (p) + 0) + e^{2 t}  (p e^t - p+1)^{n - 2}  (\frac{ d}{ dt} (2 t))) + n p ((n - 1) e^t (p e^t - p+1)^{n - 2}  (p (\frac{ d}{ dt} (e^t)) - \frac{ d}{ dt} (p) + \frac{ d}{ dt} (1)) + e^t (p e^t - p+1)^{n - 1} )
  | The derivative of 1 is zero:
= | (n - 1) n p^2 ((n - 2) e^{2 t}  (p e^t - p+1)^{n - 3}  (p (\frac{ d}{ dt} (e^t)) - \frac{ d}{ dt} (p)) + e^{2 t}  (p e^t - p+1)^{n - 2}  (\frac{ d}{ dt} (2 t))) + n p ((n - 1) e^t (p e^t - p+1)^{n - 2}  (p (\frac{ d}{ dt} (e^t)) - \frac{ d}{ dt} (p) + 0) + e^t (p e^t - p+1)^{n - 1} )
  | The derivative of e^t is e^t:
= | (n - 1) n p^2 ((n - 2) e^{2 t}  (p e^t - p+1)^{n - 3}  (p e^t - \frac{ d}{ dt} (p)) + e^{2 t}  (p e^t - p+1)^{n - 2}  (\frac{ d}{ dt} (2 t))) + n p ((n - 1) e^t (p e^t - p+1)^{n - 2}  (p (\frac{ d}{ dt} (e^t)) - \frac{ d}{ dt} (p)) + e^t (p e^t - p+1)^{n - 1} )
  | The derivative of e^t is e^t:
= | (n - 1) n p^2 ((n - 2) e^{2 t}  (p e^t - p+1)^{n - 3}  (p e^t - \frac{ d}{ dt} (p)) + e^{2 t}  (p e^t - p+1)^{n - 2}  (\frac{ d}{ dt} (2 t))) + n p ((n - 1) e^t (p e^t - p+1)^{n - 2}  (p e^t - \frac{ d}{ dt} (p)) + e^t (p e^t - p+1)^{n - 1} )
  | The derivative of p is zero:
= | (n - 1) n p^2 (e^{2 t}  (p e^t - p+1)^{n - 2}  (\frac{ d}{ dt} (2 t)) + (n - 2) e^{2 t}  (p e^t - 0) (p e^t - p+1)^{n - 3} ) + n p ((n - 1) e^t (p e^t - p+1)^{n - 2}  (p e^t - \frac{ d}{ dt} (p)) + e^t (p e^t - p+1)^{n - 1} )
  | The derivative of p is zero:
= | (n - 1) n p^2 (e^{2 t}  (p e^t - p+1)^{n - 2}  (\frac{ d}{ dt} (2 t)) + (n - 2) p e^{3 t}  (p e^t - p+1)^{n - 3} ) + n p ((n - 1) e^t (p e^t - 0) (p e^t - p+1)^{n - 2}  + e^t (p e^t - p+1)^{n - 1} )
  | Factor out constants:
= | (n - 1) n p^2 (e^{2 t}  (p e^t - p+1)^{n - 2}  (2 (\frac{ d}{ dt} (t))) + (n - 2) p e^{3 t}  (p e^t - p+1)^{n - 3} ) + n p ((n - 1) p e^{2 t}  (p e^t - p+1)^{n - 2}  + e^t (p e^t - p+1)^{n - 1} )
  | The derivative of t is 1:
= | (n - 1) n p^2 ((n - 2) p e^{3 t}  (p e^t - p+1)^{n - 3}  + 2 e^{2 t}  (p e^t - p+1)^{n - 2} ) + n p ((n - 1) p e^{2 t}  (p e^t - p+1)^{n - 2}  + e^t (p e^t - p+1)^{n - 1} )
            &= n p e^t (p (e^t  -  1)  +  1)^{n  -  3} (p^2 (n^2 e^2t  +  (1  -  3 n) e^t  +  1) 
             +  p ((3n  -  1) e^t  - 2)  +  1)\\
          \text{Hence}\\
          M'''(0) &=
      \end{align*}
      \end{homeworkSection}
    \item An exponential $(\lambda)$ random variable
      \begin{homeworkSection}{Solution}
        
      \end{homeworkSection}
  \end{enumerate}
\end{homeworkProblem}

%=============================Problem3==========================% 
\newpage
\begin{homeworkProblem}
  Simplify as much as possible $\left(\frac{ m_4}{ \sigma^4}\right) - 3$ 
  for an $n(\mu, \sigma^2)$ random variable, where here the $m_4$ is the
  4th power moment.
  \begin{homeworkSection}{Solution}
    
  \end{homeworkSection}
\end{homeworkProblem}

%=============================Problem4==========================% 
\newpage
\begin{homeworkProblem}
  Use the \emph{binomial expansion} to derive the $M(t)$ for a negagtive
  binomial $(r, p)$ random variable.
  \begin{homeworkSection}{Solution}
    
  \end{homeworkSection}
\end{homeworkProblem}
%=============================Problem5==========================% 
\newpage
\begin{homeworkProblem}
  Use 4 to verify the formulas for mean and variance given on p.385 
  (last row of the table).
  \begin{homeworkSection}{Solution}
    
  \end{homeworkSection}
\end{homeworkProblem}
  
\end{spacing}
\end{document}

\begin{comment}%==========================================================
%=============================Problemi==========================% 
\newpage
\begin{homeworkProblem}
  
  \begin{homeworkSection}{Solution}
    
  \end{homeworkSection}
\end{homeworkProblem}
%=============================Problemi==========================% 
\newpage
\begin{homeworkProblem}
  
  \begin{enumerate}[(a)]
    \item 
      \begin{homeworkSection}{Solution}
    
      \end{homeworkSection}
  \end{enumerate}
\end{homeworkProblem}

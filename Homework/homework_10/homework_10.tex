\documentclass{article}
\usepackage{hw_style}
\usepackage{enumerate}
\usepackage{graphicx}
\usepackage{verbatim}

% Homework Specific Information
\newcommand{\hmwkTitle}{Homework \#10}
\newcommand{\hmwkDueDate}{Wednesday, July 18, 2012}
\newcommand{\hmwkAuthorName}{Kurt Rudolph}%Name:
\newcommand{\hmwkNetID}{rudolph9}%your netid
\newcommand{\hmwkNotes}{}%I worked with...

\newcommand{\hmwkSubTitle}{}
\newcommand{\hmwkClass}{Math 461}
\newcommand{\hmwkClassTime}{}
\newcommand{\hmwkClassInstructor}{Kenneth B. Stolarsky}

\begin{document}
\begin{spacing}{1.1}
\maketitle

%=============================p.412 #1==========================% 
\newpage
\begin{homeworkProblem}
  {\bf Chapter 8, Exercise 1}\\
  Suppose that $X$ is a random variable with mean and variance 
  both equal to 20. What can be said about $P\{ 0 < X < 40\}$?
  \begin{homeworkSection}{Solution}
    
  \end{homeworkSection}
\end{homeworkProblem}

%=============================p.412 #2==========================% 
\newpage
\begin{homeworkProblem}
  {\bf Chapter 8, Exercise 2}\\
  From past experience, a professor knows that the test score of a 
  student taking her final examination is a random variable with 
  mean 75.
  \begin{enumerate}[(a)]
    \item Give an upper bound for the probability that a student's 
    test score will exceed 85. 
    Suppose, in addition, that the professor knows that the variance 
    of a student's test score is equal to 25.
      \begin{homeworkSection}{Solution}
        
      \end{homeworkSection}
    \item What can be said about the probability that a student 
    will score between 65 and 85?
      \begin{homeworkSection}{Solution}
        
      \end{homeworkSection}
    \item How many students would have to take the examination to 
    ensure, with probability at least .9, that the class average 
    would be within 5 of 75? Do not use the central limit theorem.
      \begin{homeworkSection}{Solution}
        
      \end{homeworkSection}
  \end{enumerate}
\end{homeworkProblem}
  

%=============================p.413 #3==========================% 
\newpage
\begin{homeworkProblem}
  {\bf Chapter 8, Exercise 3}\\
  Use the central limit theorem to solve part (c) of Problem 2.
  \begin{homeworkSection}{Solution}
    
  \end{homeworkSection}
\end{homeworkProblem}

%=============================p.413 #4==========================% 
\newpage
\begin{homeworkProblem}
  {\bf Chapter 8, Exercise 4}\\
  Let $X_1, \dots, X_{20}$ be independent Poisson random variables 
  with mean 1.
  \begin{enumerate}[(a)]
    \item Use the Markov inequality to obtain a bound on
      \[P\left\{ \sum\limits_1^{20} X_i > 15 \right\}\]
      \begin{homeworkSection}{Solution}
        
      \end{homeworkSection}
    \item Use the central limit theorem to approximate
      \[P\left\{ \sum\limits_1^{20} X_i > 15 \right\}\]
      \begin{homeworkSection}{Solution}
        
      \end{homeworkSection}
  \end{enumerate}
\end{homeworkProblem}

%=============================p.413 #5==========================% 
\newpage
\begin{homeworkProblem}
  {\bf Chapter 8, Exercise 5}\\
  Fifty numbers are rounded off to the nearest integer and then summed. 
  If the individual roundoff errors are uniformly distributed over 
  $(−.5, .5)$, approximate the probability that the resultant sum differs 
  from the exact sum by more than 3.
  \begin{homeworkSection}{Solution}
    
  \end{homeworkSection}
\end{homeworkProblem}

%=============================p.413 #6==========================% 
\newpage
\begin{homeworkProblem}
  {\bf Chapter 8, Exercise 6}\\
  A die is continually rolled until the total sum of all rolls 
  exceeds 300. Approximate the probability that at least 80 rolls 
  are necessary.
  \begin{homeworkSection}{Solution}
    
  \end{homeworkSection}
\end{homeworkProblem}

%=============================p.414 #1==========================% 
\newpage
\begin{homeworkProblem}
  {\bf Chapter 8, Theoretical Exercise 1}\\
  If X has variance $\sigma^2$ , then $\sigma$ , the positive square root 
  of the variance, is called the standard deviation. If $X$ has mean $\mu$ 
  and standard deviation $\sigma$, show that
    \[P\{ |X - \mu| \ge k \sigma \} \le \frac{ 1}{ k^2}\]
  \begin{homeworkSection}{Solution}
    
  \end{homeworkSection}
\end{homeworkProblem}

%=============================p.414 #2==========================% 
\newpage
\begin{homeworkProblem}
  {\bf Chapter 8, Theoretical Exercise 2}\\
  If $X$ has mean $\mu$ and standard deviation $\sigma$, the ratio 
  $r \equiv |\mu|/\sigma$ is called the measurement \emph{signal-to-noise ratio} 
  of $X$. The idea is that $X$ can be expressed as $X = \mu + (X − \mu)$, with $\mu$ 
  representing the signal and $X − \mu$ the noise. If we define 
  $|(X − \mu)/|\mu| \equiv D$ as the relative deviation of $X$ from its signal 
  (or mean) $\mu$, show that, for $\alpha > 0$,
    \[P\{ D \le \alpha\} \ge 1 - \frac{ 1}{ r^2 \alpha^2}\]
  \begin{homeworkSection}{Solution}
    
  \end{homeworkSection}
\end{homeworkProblem}

%=============================p.414 #3==========================% 
\newpage
\begin{homeworkProblem}
  {\bf Chapter 8, Theoretical Exercise 3abc}\\
  Compute the measurement signal-to-noise ratio--
  that is, $|\mu|/\sigma$, where $\mu = E[ X]$ and $\sigma^2 = Var( X)$--
  of the following random variables
  \begin{enumerate}[(a)]
    \item Poisson with mean $\lambda$
      \begin{homeworkSection}{Solution}
      
      \end{homeworkSection}
    \item Binomial with parameters $n$ and $p$
      \begin{homeworkSection}{Solution}
      
      \end{homeworkSection}
    \item Geometric with mean $1/p$
      \begin{homeworkSection}{Solution}
      
      \end{homeworkSection}
  \end{enumerate}
\end{homeworkProblem}

%=============================p.415 #8==========================% 
\newpage
\begin{homeworkProblem}
  {\bf Chapter 8, Theoretical Exercise 8}\\
  Explain why a gamma random variable with parameters $(t, \lambda)$ has 
  an approximately normal distribution when $t$ is large.
  \begin{homeworkSection}{Solution}
    
  \end{homeworkSection}
\end{homeworkProblem}


\end{spacing}
\end{document}

\begin{comment}%==========================================================
* p.412
  * 1
  * 2
  * 3
  * 4
  * 5
  * 6
* p.414
  * 1
  * 2
  * 3abc
  * 8
  * DNHI
      * 4
      * 5

%=============================p.412 #1==========================% 
\newpage
\begin{homeworkProblem}
  
  \begin{homeworkSection}{Solution}
    
  \end{homeworkSection}
\end{homeworkProblem}
%=============================Problemi==========================% 
\newpage
\begin{homeworkProblem}
  
  \begin{enumerate}[(a)]
    \item 
      \begin{homeworkSection}{Solution}
    
      \end{homeworkSection}
  \end{enumerate}
\end{homeworkProblem}
\end{comment}

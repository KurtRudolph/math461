\documentclass{article}
\usepackage{hw_style}
\usepackage{enumerate}
\usepackage{graphicx}
\usepackage{verbatim}
\usepackage{amsmath}
\usepackage{mathtools}

% Homework Specific Information
\newcommand{\hmwkTitle}{Homework \#10}
\newcommand{\hmwkDueDate}{Wednesday, July 18, 2012}
\newcommand{\hmwkAuthorName}{Kurt Rudolph}%Name:
\newcommand{\hmwkNetID}{rudolph9}%your netid
\newcommand{\hmwkNotes}{}%I worked with...

\newcommand{\hmwkSubTitle}{}
\newcommand{\hmwkClass}{Math 461}
\newcommand{\hmwkClassTime}{}
\newcommand{\hmwkClassInstructor}{Kenneth B. Stolarsky}

\begin{document}
\begin{spacing}{1.1}
\maketitle

%=============================p.412 #1==========================% 
\newpage
\begin{homeworkProblem}
  {\bf Chapter 8, Exercise 1}\\
  Suppose that $X$ is a random variable with mean and variance 
  both equal to 20. What can be said about $P\{ 0 < X < 40\}$?
  \begin{homeworkSection}{Definitions}
    {\bf Chebyshev's Inequality}\\
      \[P\{ |X - \mu| \ge k\} \le \frac{ \sigma^2}{ k^2}\]
  \end{homeworkSection}
  \begin{homeworkSection}{Solution}
    \begin{align*}
      \text{Given}\\
      \mu &= 20\\
      \sigma^2 &= 20\\
      \text{we find}\\
      P\{ 0 < X < 40\} &= P\{ -20 < X - 20 < 20\}\\
      &= 1 - P\{|X - 20| > 20\} \\
      \text{by Chebyshev's Inequality}\\
      P\{|X - 20| > 20\} &\le \frac{ 20}{ 20^2}\\
      &= 0.05\\
      \text{hence}\\
      1 - P\{|X - 20| > 20\} &> 1 - 0.05\\
      &= 0.95\\
      \text{Therefore}\\
      P\{ 0 < X < 40\} &> 0.95\\
    \end{align*}
  \end{homeworkSection}
\end{homeworkProblem}

%=============================p.412 #2==========================% 
\newpage
\begin{homeworkProblem}
  {\bf Chapter 8, Exercise 2}\\
  From past experience, a professor knows that the test score of a 
  student taking her final examination is a random variable with 
  mean 75.
  \begin{homeworkSection}{Definitions}
    {\bf Markov's inequality}
    If $X$ is a random variable that takes only nonnegative values, 
    then, for any value $a > 0$,
      \[P\{ X \ge a\} \le \frac{ E[ X]}{ a}\]
    {\bf Chebyshev inequality}
    If $X$ is a random variable that takes only nonnegative values, 
    then, for any value
      \[P\{ |X - \mu| \ge k \sigma \} \le \frac{ 1}{ k^2}\]
    {\bf One-sided Chebyshev Corollary}
      \[P\{ X \ge \mu + a\} \le \frac{ \sigma^2}{ \sigma^2 + a^2}\]
  \end{homeworkSection}
  \begin{enumerate}[(a)]
    \item Give an upper bound for the probability that a student's 
    test score will exceed 85. 
    Suppose, in addition, that the professor knows that the variance 
    of a student's test score is equal to 25.
      \begin{homeworkSection}{Solution}
        \begin{align*}
          \text{Given}\\
          \mu &= 75\\
          \sigma^2 &= 25\\
          \text{by one-sided Chebyshev Corollary}\\
          P\{ X \ge 85\} &= P\{ X \ge 75 + 10\}\\
          &\le \frac{ 25}{ 125}\\
          &= \frac{ 1}{ 5}\\
        \end{align*}
      \end{homeworkSection}
    \item What can be said about the probability that a student 
    will score between 65 and 85?
      \begin{homeworkSection}{Solution}
        \begin{align*}
          \text{Given}\\
          \mu &= 75\\
          \sigma = 5\\
          \text{by Chebyshev Inequality}\\
          k &= 2\\
          P\{ |X - 75| \ge (2)(5)\} &\le \frac{ 1}{ 4}\\
          &= 0.25\\
          \text{Therefore}\\
          P\{ |X - 75| \le 10\} &= 1 - P\{ |X - 75| \ge 10\}\\
          &= 1 - 0.25\\
          &= 0.75\\
        \end{align*}
      \end{homeworkSection}
    \item How many students would have to take the examination to 
    ensure, with probability at least .9, that the class average 
    would be within 5 of 75? Do not use the central limit theorem.
      \begin{homeworkSection}{Solution}
        I don't know.
      \end{homeworkSection}
  \end{enumerate}
\end{homeworkProblem}
  

%=============================p.413 #3==========================% 
\newpage
\begin{homeworkProblem}
  {\bf Chapter 8, Exercise 3}\\
  Use the central limit theorem to solve part (c) of Problem 2.
  \begin{homeworkSection}{Solution}
     \begin{align*}
        P\{ \left|\frac{ \frac{ 1}{ n}\sum\limits_{i = 1}^{n} X_i - 75}{ \frac{ 5}{ \sqrt{ n}}}\right| \le \frac{ 5}{ \frac{ 5}{ \sqrt{ n}}}\} &\ge 0.9\\
        \text{solving for $n$}\\
        P\{ \left|\frac{ \frac{ 1}{ n}\sum\limits_{i = 1}^{n} X_i - 75}{ \frac{ 5}{ \sqrt{ n}}}\right| \le \frac{ 5}{ \frac{ 5}{ \sqrt{ n}}}\}
        &= P\{ \left|\frac{ \frac{ 1}{ n}\sum\limits_{i = 1}^{n} X_i - 75}{ \frac{ 5}{ \sqrt{ n}}}\right| \le \sqrt{ n}\}\\
        &= 1 - 2P\{ \frac{ \frac{ 1}{ n}\sum\limits_{i = 1}^{n} X_i - 75}{ \frac{ 5}{ \sqrt{ n}}} \le \sqrt{ n}\}\\
        &= 1 - 2 \Phi(-sqrt{n})\\
        \text{hence}\\
        1 - 2 \Phi(-sqrt{n}) &\ge 0.9\\
        \text{therefore}\\
        n &> (-\Phi^{-1}(0.05))^2 = 2.7055
      \end{align*}
  \end{homeworkSection}
\end{homeworkProblem}

%=============================p.413 #4==========================% 
\newpage
\begin{homeworkProblem}
  {\bf Chapter 8, Exercise 4}\\
  Let $X_1, \dots, X_{20}$ be independent Poisson random variables 
  with mean 1.
  \begin{enumerate}[(a)]
    \item Use the Markov inequality to obtain a bound on
      \[P\left\{ \sum\limits_1^{20} X_i > 15 \right\}\]
      \begin{homeworkSection}{Solution}
        \begin{align*}
          \text{Let}\\
          X &= \sum\limits_{i = 1}^{20} X_i\\
          \text{hence}\\
          E[X] &= \sum\limits_{i = 1}^{20} E[ X_i]\\ 
          &= 20
          \text{Therefore}
          P\{ X \ge 15\} \le \frac{ 4}{ 3}\\
        \end{align*}
      \end{homeworkSection}
    \item Use the central limit theorem to approximate
      \[P\left\{ \sum\limits_1^{20} X_i > 15 \right\}\]
      \begin{homeworkSection}{Solution}
        \begin{align*}
          \text{Given}\\
          \mu &= 1\\
          \text{applying the central limit theorem}\\
          P\{ \frac{ \sum\limits_{i = 1}^{20} X_i - 20}{ \sqrt{ 20}} \ge \frac{ 15 - 20}{ \sqrt{20}}\}\\
          &= 1 - P\{ Z < -\frac{ 5}{ \sqrt{ 20}}\}\\
          &= 0.8682
        \end{align*}
      \end{homeworkSection}
  \end{enumerate}
\end{homeworkProblem}

%=============================p.413 #5==========================% 
\newpage
\begin{homeworkProblem}
  {\bf Chapter 8, Exercise 5}\\
  Fifty numbers are rounded off to the nearest integer and then summed. 
  If the individual roundoff errors are uniformly distributed over 
  $(−.5, .5)$, approximate the probability that the resultant sum differs 
  from the exact sum by more than 3.
  \begin{homeworkSection}{Solution}
    \begin{align*}
      \text{Let $R$ be the rounded sum.}\\
      \text{Let $X$ be the exact sum.}\\
      P\{ |X - R| > 3 \} &= P\{ \left|\sum\limits_{i = 1}^{50} (X_i - R_i)\right| > 3\}\\
      \text{Applying the central limit theorem}\\
      \sigma^2 &= \frac{ (0.5 - (- 0.5))^2}{ 12} \\
      &= \frac{ 1}{ 12}\\
      P\{ \left|\sum\limits_{i = 1}^{50} (X_i - R_i)\right| > 3\}
      &= P\{ \left|\frac{ \sum\limits_{i = 1}^{50} (X_i - R_i)}{ \frac{ 50}{ \sqrt{12}}}\right| > \frac{ 3}{ \frac{ 50}{ \sqrt{12}}}\\
      &= 2P\{ \left|\frac{ \sum\limits_{i = 1}^{50} (X_i - R_i)}{ \frac{ 50}{ \sqrt{12}}}\right| < \frac{ -3}{ \frac{ 50}{ \sqrt{12}}}\\
      &= 2 \phi\left( \frac{ -3}{ \frac{ 50}{ \sqrt{12}}}\right)\\
      &= 0.8353\\
    \end{align*}
  \end{homeworkSection}
\end{homeworkProblem}

%=============================p.413 #6==========================% 
\newpage
\begin{homeworkProblem}
  {\bf Chapter 8, Exercise 6}\\
  A die is continually rolled until the total sum of all rolls 
  exceeds 300. Approximate the probability that at least 80 rolls 
  are necessary.
  \begin{homeworkSection}{Solution}
    \begin{align*}
      \text{Given}\\
      p(x) &= \frac{ 1}{ 6}\\
      \text{since the sum of $n$ die rolls is}\\
      X &= \sum\limits_{i = 1}^n X_i\\
      \text{where $X_i \in \{1,2,3,4,5,6\}$}\\
      \text{then}\\
      \mu &= E[ X] \\
        &= E\left[ \sum\limits_{i = 1}^n E[ X_i]\right] \\
        &= n \frac{ 1}{ 6} (1 + 2 + 3 + 4 + 5 + 6)\\
        &= \frac{ 7n}{ 2}\\
      \text{and it follows that}\\
      Var( X) &= n Var(X_i) \\
        &= n (E[X_i^2] - E[X_i]^2\\
        &= n \left(\frac{ 1}{ 6} (1 + 2 + 3 + 4 + 5 + 6) - \left(\frac{ 7}{ 2}\right)^2\right)\\
        &= 2.916 n\\
      \text{Applying the central limit theorm to our desired probability}\\
      P\{ X > 300\} 
        &= P\left\{ \frac{ X - \frac{ 7 n}{ 2}}{ \sqrt{ 2.916} \sqrt{ n}} 
          > \frac{ 300 - \frac{ 7 n}{ 2}}{ \sqrt{2.916} \sqrt{ n}}\right\}\\
      \text{letting $n = 80$}\\
        &= P\left\{ \frac{ X - 280}{ \sqrt{ 2.916} \sqrt{ 80}} 
          > \frac{ 300 - 280}{ \sqrt{2.916} \sqrt{ 80}}\right\}\\
        &= 1 - P\{Z < 1.309\} = 1 - \Phi(1.309) = 0.0953
    \end{align*}
  \end{homeworkSection}
\end{homeworkProblem}

%=============================p.414 #1==========================% 
\newpage
\begin{homeworkProblem}
  {\bf Chapter 8, Theoretical Exercise 1}\\
  If X has variance $\sigma^2$ , then $\sigma$ , the positive square root 
  of the variance, is called the standard deviation. If $X$ has mean $\mu$ 
  and standard deviation $\sigma$, show that
    \[P\{ |X - \mu| \ge k \sigma \} \le \frac{ 1}{ k^2}\]
  \begin{homeworkSection}{Definitions}
    {\bf Chebyshev's inequality}
      If $X$ is a random variable with finite mean $\mu$ and variance $\sigma^2$, 
      then, for any value $k > 0$,
      \[P\{ |X - \mu| \ge k\} \le \frac{ \sigma^2}{ k^2}\]
  \end{homeworkSection}
  \begin{homeworkSection}{Solution}
    \begin{align*}
      \text{Building off Chebyshev's inequality, let}\\
      k &= \sigma \hat k\\
      \text{which gives us}\\
      P\{ |X - \mu| \ge \sigma \hat k\} &\le \frac{ \sigma^2}{ (\sigma \hat k)^2}\\
      &= \frac{ 1}{\hat k^2}
    \end{align*}
  \end{homeworkSection}
\end{homeworkProblem}

%=============================p.414 #2==========================% 
\newpage
\begin{homeworkProblem}
  {\bf Chapter 8, Theoretical Exercise 2}\\
  If $X$ has mean $\mu$ and standard deviation $\sigma$, the ratio 
  $r \equiv |\mu|/\sigma$ is called the measurement \emph{signal-to-noise ratio} 
  of $X$. The idea is that $X$ can be expressed as $X = \mu + (X − \mu)$, with $\mu$ 
  representing the signal and $X − \mu$ the noise. If we define 
  $|(X − \mu)/\mu| \equiv D$ as the relative deviation of $X$ from its signal 
  (or mean) $\mu$, show that, for $\alpha > 0$,
    \[P\{ D \le \alpha\} \ge 1 - \frac{ 1}{ r^2 \alpha^2}\]
  \begin{homeworkSection}{Solution}
    \begin{align*}
      \text{Substituing for $D$ and $\alpha$}\\
      P\left\{ \left|\frac{ X \mu}{ \mu}\right| \le \alpha\right\} &\\
      P\left\{ \left| X\right| \le \alpha\right\} 
        &\ge 1 - \frac{ 1}{ \left(\frac{ |\mu|}{ \sigma}\right)^2 \alpha^2}\\
        &= 1 - \frac{ \sigma^2}{ \mu^2 \alpha^2}\\
      \text{Letting k = $\mu \alpha$}\\
      P\left\{ \left| X\right| \le \frac{ k}{ \mu}\right\} 
        &\ge 1 - \frac{ \sigma^2}{ k^2}\\ 
      P\left\{ \left|X \mu\right| \le k\right\} 
        &\ge 1 - \frac{ \sigma^2}{ k^2}\\ 
      \text{Since $X = \mu + (X \mu) \Rightarrow X - \mu = X\mu$}\\
      P\left\{ \left|X - \mu\right| \le k\right\} 
        &\ge 1 - \frac{ \sigma^2}{ k^2}\\ 
      \text{Which is the follows the Chebyshev's inequality}
    \end{align*}
  \end{homeworkSection}
\end{homeworkProblem}

%=============================p.414 #3==========================% 
\newpage
\begin{homeworkProblem}
  {\bf Chapter 8, Theoretical Exercise 3abc}\\
  Compute the measurement signal-to-noise ratio--
  that is, $|\mu|/\sigma$, where $\mu = E[ X]$ and $\sigma^2 = Var( X)$--
  of the following random variables
  \begin{enumerate}[(a)]
    \item Poisson with mean $\lambda$
      \begin{homeworkSection}{Solution}
        \[\frac{ |\mu|}{ \sigma} = \frac{ |\lambda|}{ \sqrt{\lambda}} = \sqrt{ |\lambda|}\]
      \end{homeworkSection}
    \item Binomial with parameters $n$ and $p$
      \begin{homeworkSection}{Solution}
        \[\frac{ |\mu|}{ \sigma} 
          = \frac{ |np|}{ \sqrt{ n p (1 - p)}} 
          = \frac{ \sqrt{ n p (1 - p)}}{ 1 - p}\]
      \end{homeworkSection}
    \item Geometric with mean $1/p$
      \begin{homeworkSection}{Solution}
        \[\frac{ |\mu|}{ \sigma} 
          = \frac{ \left|\frac{ 1}{ p}\right|}{ \sqrt{ \frac{ 1 - p}{ p^2}}} 
          = \frac{ \sqrt{ p^2}}{ |p| \sqrt{ 1 - p}} 
          = (1 - p)^{-1/2}\]
      
      \end{homeworkSection}
  \end{enumerate}
\end{homeworkProblem}

%=============================p.415 #8==========================% 
\newpage
\begin{homeworkProblem}
  {\bf Chapter 8, Theoretical Exercise 8}\\
  Explain why a gamma random variable with parameters $(t, \lambda)$ has 
  an approximately normal distribution when $t$ is large.
  \begin{homeworkSection}{Definitions}
    {\bf Gamma Random Variable}\\
    A random variable is said to have a gamma distribution with parameters 
    $(\alpha, \lambda), \lambda > 0, \alpha > 0$, if its density function 
    is given by
      \[f(x) = 
        \left\{ \begin{array}{l l}
          \frac{ \lambda e^{-\lambda x}(\lambda x)^{\alpha - 1}}{ \Gamma( \alpha)} & x \ge 0\\
          0 & x < 0
        \end{array}\right.
      \]
    where $\Gamma( \alpha)$, called the \emph{gamma function}, is defined as
      \[\Gamma( \alpha) 
        = \int\limits_0^\infty e^{-y} y^{\alpha - 1} dy 
        = (\alpha - 1)\Gamma( \alpha -1)\]
    {\bf Normal Random Variable}\\
    We say that $X$ is a normal random variable, or simply that $X$ is normally 
    distributed, with parameters $\mu$ and $\sigma^2$ if the density of $X$ is given by
      \[\begin{array}{l l} f( x) = \frac{ 1}{ \sqrt{ 2 \pi} \sigma} e^{-(x - \mu)^2/ 2 \sigma^2} & -\infty < x < \infty \end{array}\]
  \end{homeworkSection}
  \begin{homeworkSection}{Solution}
    The skewness of the \emph{gamma distrobution} depends only on the 
    first perameter, in the context of this problem $t$.  Hence, as 
    $t \rightarrow \infty$ we see the \emph{gamma distrobution} approaches 
    a \emph{normal distrobution}.
    
  \end{homeworkSection}
\end{homeworkProblem}


\end{spacing}
\end{document}

\begin{comment}%==========================================================
* p.412
  * 1
  * 2
  * 3
  * 4
  * 5
  * 6
* p.414
  * 1
  * 2
  * 3abc
  * 8
  * DNHI
      * 4
      * 5

%=============================p.412 #1==========================% 
\newpage
\begin{homeworkProblem}
  
  \begin{homeworkSection}{Solution}
    
  \end{homeworkSection}
\end{homeworkProblem}
%=============================Problemi==========================% 
\newpage
\begin{homeworkProblem}
  
  \begin{enumerate}[(a)]
    \item 
      \begin{homeworkSection}{Solution}
    
      \end{homeworkSection}
  \end{enumerate}
\end{homeworkProblem}
\end{comment}

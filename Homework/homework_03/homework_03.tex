\documentclass{article}
\usepackage{hw_style}
\usepackage{enumerate}
\usepackage{graphicx}
\usepackage{verbatim}

% Homework Specific Information
\newcommand{\hmwkTitle}{Homework \#3}
\newcommand{\hmwkDueDate}{Tuesday, June 19th at 9:00AM}
\newcommand{\hmwkAuthorName}{Kurt Rudolph}%Name:
\newcommand{\hmwkNetID}{rudolph9}%your netid
\newcommand{\hmwkNotes}{}%I worked with...

\newcommand{\hmwkSubTitle}{}
\newcommand{\hmwkClass}{Math 461}
\newcommand{\hmwkClassTime}{}
\newcommand{\hmwkClassInstructor}{Kenneth B. Stolarsky}

\begin{document}
\begin{spacing}{1.1}
\maketitle
%=============================pg55 problem11==========================%	
\newpage
\begin{homeworkProblem}
  {\bf Chapter 2, Theoretical Exercise 11} 

  If $P( E) = .9$ and $P( F) = .8$, show that $P( EF) \ge .7$. In general, prove Bonferroni’s inequality, namely,

  \[P( EF) \ge P( E) + P( F) - 1\]

  \begin{homeworkSection}{Bonferroni Inequality}
    Let $P( E_i)$ be the probability that $E_i$ is true, and $P\left( \bigcup\limits_{i = 1}^{n}{ E_i}\right)$ be the probability 
    that at least one of $E_1, E_2, \dots, E_n$ is true.  The the \emph{Bonferroni Inequality}, also knowon as \emph{Boole's Inequality} states

    \[P\left( \bigcup\limits_{i = 1}^{n}{ E_i}\right) \le \sum\limits_{i = 1}^{n}{ P( E_i)}\]
  \end{homeworkSection}	

  \begin{homeworkSection}{Inclusion-Exclusion Principle}
    Given from Proposition 4.2
      \[P( E \cup F) = P( E) + P( F) - P( EF)\]
    The more general form
      \[\left|\bigcup\limits_{i = 1}^{n}{ E_i}\right| \le \sum\limits_{i = 1}^{n}{ \left|E_i\right|}\]
  \end{homeworkSection}

	\begin{homeworkSection}{Solution}
    Since $P( E \cup F) \le 1$ it follows that 
      \[1 \ge P( E) + P( F) - P( EF)\]
      \[P( EF) \ge P( E) + P( F) - 1\]
    This follows Bonferroni Inequality and show that
      \[P( EF) \ge 0.9 + 0.8 - 1 = 0.7\]
    
	\end{homeworkSection}
\end{homeworkProblem}

%=============================pg55 problem16==========================%	
\newpage
\begin{homeworkProblem}
  {\bf Chapter 2, Theoretical Exercise 16}
  
  Use induction to generalize Bonferroni’s inequality to $n$ events. That is, show that
    \[P( E_1 E_2 \cdots E_n) \ge P( E_1) + \cdots + P( E_n) - (n - 1)\]

  \begin{homeworkSection}{Bonferroni Inequality}
    Let $P( E_i)$ be the probability that $E_i$ is true, and $P\left( \bigcup\limits_{i = 1}^{n}{ E_i}\right)$ be the probability 
    that at least one of $E_1, E_2, \dots, E_n$ is true.  The the \emph{Bonferroni Inequality}, also knowon as \emph{Boole's Inequality} states

    \[P\left( \bigcup\limits_{i = 1}^{n}{ E_i}\right) \le \sum\limits_{i = 1}^{n}{ P( E_i)}\]
  \end{homeworkSection}	

  \begin{homeworkSection}{Inclusion-Exclusion Principle}
    Given from Proposition 4.2
      \[P( E \cup F) = P( E) + P( F) - P( EF)\]
    The more general form
      \[\left|\bigcup\limits_{i = 1}^{n}{ E_i}\right| \le \sum\limits_{i = 1}^{n}{ \left|E_i\right|}\]
  \end{homeworkSection}

	\begin{homeworkSection}{Solution}
    As shown in {\bf Chapter 2, Theoretical Exercise 11}, $P( EF) \ge P( E) + P( F) - 1$. 
    Using this inequality for two sets as our base case we must now consider the case where
    we have $n + 1$ sets.  A similar proof follows
      \[P( E_1 E_2 E_3 \cdots E_{n + 1}) \ge P( E_1) + P(E_2 E_3 \cdots E_{n + 1}) - 1\]
      \[P( E_1 E_2 E_3 \cdots E_{n + 1}) \ge P( E_1) + P( E_2) + P(E_3 E_4 \cdots E_{n + 1}) - 2\]
      \[P( E_1 E_2 E_3 \cdots E_{n + 1}) \ge \cdots\]
      \[P( E_1 E_2 E_3 \cdots E_{n + 1}) \ge P( E_1) + P( E_2) + \cdots P( E_{n + 1}) - (n)\]

    Since our basecase holds for $n = 2$, by inductive proof therefore 
      \[P( E_1 E_2 E_3 \cdots E_{n + 1}) \ge P( E_1) + P( E_2) + \cdots P( E_{n}) - (n - 1)\]

	\end{homeworkSection}
\end{homeworkProblem}

%=============================pg102 problem2==========================%	
\newpage
\begin{homeworkProblem}
  {\bf Chapter 3, Problem 2	}
  
  If two fair dice are rolled, what is the conditional probability that the first one lands 
  on 6 given that the sum of the dice is $i$? Compute for all values of $i$ between 2 and 12.

  \begin{homeworkSection}{Solution}
    Let
    \begin{itemize}
      \item $F_6$ be the event the first die lands on a 6.
      \item $E_i$ the event the sum of the dice is $i$.
    \end{itemize}
    \[P( F_6|E_i) = \frac{ P( F_6 \cap E_i)}{ P( E_i)}\]
    \[P( F_6|E_2) = \frac{ P( F_6 \cap E_2)}{ P( E_2)} = \emptyset\]
    \[P( F_6|E_3) = \frac{ P( F_6 \cap E_3)}{ P( E_3)} = \emptyset\]
    \[P( F_6|E_4) = \frac{ P( F_6 \cap E_4)}{ P( E_4)} = \emptyset\]
    \[P( F_6|E_5) = \frac{ P( F_6 \cap E_5)}{ P( E_5)} = \emptyset\]
    \[P( F_6|E_6) = \frac{ P( F_6 \cap E_6)}{ P( E_6)} = \emptyset\]
    \[P( F_6|E_7) = \frac{ P( F_6 \cap E_7)}{ P( E_7)} = \frac{ \frac{ 1}{ 36}}{ \frac{ 6}{ 36}} = \frac{ 1}{ 6}\]
    \[P( F_6|E_8) = \frac{ P( F_6 \cap E_8)}{ P( E_8)} = \frac{ \frac{ 1}{ 36}}{ \frac{ 5}{ 36}} = \frac{ 1}{ 5}\]
    \[P( F_6|E_9) = \frac{ P( F_6 \cap E_9)}{ P( E_9)} = \frac{ \frac{ 1}{ 36}}{ \frac{ 4}{ 36}} = \frac{ 1}{ 4}\]
    \[P( F_6|E_10) = \frac{ P( F_6 \cap E_10)}{ P( E_10)} = \frac{ \frac{ 1}{ 36}}{ \frac{ 3}{ 36}} = \frac{ 1}{ 3}\]
    \[P( F_6|E_11) = \frac{ P( F_6 \cap E_11)}{ P( E_11)} = \frac{ \frac{ 1}{ 36}}{ \frac{ 2}{ 36}} = \frac{ 1}{ 2}\]
    \[P( F_6|E_12) = \frac{ P( F_6 \cap E_12)}{ P( E_12)} = \frac{ \frac{ 1}{ 36}}{ \frac{ 1}{ 36}} = \frac{ 1}{ 1} = 1\]
  \end{homeworkSection}
\end{homeworkProblem}
%=============================pg102 Problem5==========================%	
\newpage
\begin{homeworkProblem}
  {\bf Chapter 3, Problem 5}
  An urn contains 6 white and 9 black balls. If 4 balls are to be randomly selected without replacement, 
  what is the probability that the first 2 selected are white and the last 2 black?
  \begin{homeworkSection}{Solution}
    Let
    \begin{itemize}
      \item $W$ be the event the first two balls picked are white.
      \item $B$ Be the event the second two balls picked are black.
    \end{itemize}

    \[P( W) = \frac{ {6 \choose 2}}{ {15 \choose 2}}\]
    \[P( B|W) = \frac{ {9 \choose 2}}{ {13 \choose 2}}\]

    Therefore 
    \[P( W \cap B) = \frac{ 15}{ 105} \frac{ 36}{ 78} = \frac{ 6}{ 91}\]

  \end{homeworkSection}
\end{homeworkProblem}
%=============================pg102 problem11==========================%	
\newpage
\begin{homeworkProblem}
  {\bf Chapter 3, Problem 11}
  Two cards are randomly chosen without replacement from an ordinary deck of 52 cards. 
  Let $B$ be the event that both cards are aces, let $A_s$ be the event that the ace 
  of spades is chosen, and let $A$ be the event that at least one ace is chosen. Find
  \begin{enumerate}[(a)]
    \item $P( B|A_s)$
      \begin{homeworkSection}{Solution}
        \[P( B|A_s) = \frac{ P( B \cap A_s)}{ P( A_s)} = \frac{ \frac{ 3}{ {52 \choose 2}}}{ \frac{ 51}{ {52 \choose 2}}} = \frac{ 1}{ 17}\]
      \end{homeworkSection}
    \item $P( B|A)$
      \begin{homeworkSection}{Solution}
        \[P( B|A) = \frac{ P( B \cap A)}{ P( A)}\]
        Since $ A \subset B$
        \[\frac{ P( B \cap A)}{ P( A)} = \frac{ P( B)}{ P( A)} = \frac{ \frac{ {4 \choose 2}}{ {52 \choose 2}}}{ 1 - \frac{ {48 \choose 2}}{ {52 \choose 2}}} = \frac{ 1}{ 33}\]
      \end{homeworkSection}
  \end{enumerate}
\end{homeworkProblem}
%============================pg102 problem13==========================%	
\newpage
\begin{homeworkProblem}
  {\bf Chapter 3 Problem 13}
  
  Suppose that an ordinary deck of 52 cards (which contains 4 aces) is randomly 
  divided into 4 hands of 13 cards each. We are interested in determining $p$, 
  the probability that each hand has an ace. Let $E_i$ be the event that the 
  $i$th hand has exactly one ace. Determine $p = P( E_1 E_2 E_3 E_4)$ 
  by using the multiplication rule.
  
  \begin{homeworkSection}{Solution}
    \[p = P( E_1 E_2 E_3 E_4) \]
    \[= P( E_1) P( E_2 E_3 E_4|E_1)\]
    \[= P( E_1) P( E_2|E_1) P( E_3 E_4|E_1 E_2)\]
    \[= P( E_1) P( E_2|E_1) P( E_3|E_1 E_2) P( E_4|E_1 E_2 E_3)\]
    \[ = \frac{ {4 \choose 1} {48 \choose 12}}{ {52 \choose 13}} \frac{ {3 \choose 1} {24 \choose 12}}{ {39 \choose 13}} \frac{ {2 \choose 1} {24 \choose 12}}{ {26 \choose 13}} \frac{ {1 \choose 1} {12 \choose 12}}{ {13 \choose 13}}\]
        \[= 0.1055\]
  \end{homeworkSection}
\end{homeworkProblem}
%============================pg110 problem1==========================%	
\newpage
\begin{homeworkProblem}
  Show that if $P( A) > 0$, then
    \[P(AB|A) \ge P( AB|A \cup B)\]

  \begin{homeworkSection}{Solution}

    \[P(AB|A) = \frac{ P( AB \cap A)}{ P( A)} = \frac{ P( AB)}{ P( A)}\]
    \[P( AB|A \cup B) = \frac{ P( AB \cap (A \cup B))}{ P( A \cup B)} = \frac{ P( A \cap B)}{ P( A \cup B)}\]

    \[ A \cup B \supseteq A \Rightarrow P( A \cup B) \ge P( A)\]

    Therefore,
      \[\frac{ P( AB)}{ P( A)} \ge \frac{ P( AB)}{ P( A \cup B)} \equiv P(AB|A) \ge P( AB|A \cup B)\]
    
  \end{homeworkSection}
\end{homeworkProblem}
%=============================pg102 Problem2==========================%	
\newpage
\begin{homeworkProblem}
  Let $A \subset B$. Express the following probabilities as simply as possible:
  \begin{itemize}
    \item $P( A|B)$
      \begin{homeworkSection}{Solution}
        \[P( A|B) = \frac{ P( AB)}{ P(B)}\]
        Since $A \subset B$, 
        \[P( A|B) = \frac{ P( AB)}{ P(B)} = \frac{ P( A)}{ P( B)}\]
     \end{homeworkSection}
    \item $P( A|B^c)$
      \begin{homeworkSection}{Solution}
        \[P( A|B^c) = \frac{ P( AB^c)}{ P( B^c}\]
        Since,
          \[A \subset B \Rightarrow AB^c = \emptyset\]
        Therefore,
        \[P( A|B^c) = \frac{ P( AB^c)}{ P( B^c} = \frac{ P( \emptyset)}{ P( B^c} = 0\]
      \end{homeworkSection}
    \item $P( B|A)$
      \begin{homeworkSection}{Solution}
        \[P( B|A) = \frac{ P( BA)}{ P( A}\]
        Since,
          \[A \subset B \Rightarrow BA = A\]
        Therefore,
        \[P( B|A) = \frac{ P( BA)}{ P( A} = \frac{ P( A)}{ P( A)} = 1\]
      \end{homeworkSection}
    \item $P( B|A^c)$
      \begin{homeworkSection}{Solution}
        \[P( B|A^c) = \frac{ P( BA^c)}{ P( A^c}\]
        Since,
          \[A \subset B \Rightarrow BA = A\]
        Therefore, we can not simplify the probability any further.
      \end{homeworkSection}
  \end{itemize}
\end{homeworkProblem}

\end{spacing}
\end{document}





\begin{comment}%==========================================================
%=============================Problemi==========================%	
\newpage
\begin{homeworkProblem}

  \begin{homeworkSection}{Solution}
    
  \end{homeworkSection}
\end{homeworkProblem}
%=============================Problemi==========================%	
\begin{homeworkProblem}
	
	\begin{enumerate}[(a)]
		\item 
			\begin{homeworkSection}{Solution}
		
			\end{homeworkSection}
	\end{enumerate}
\end{homeworkProblem}
%=============================Problemi==========================%	
\begin{homeworkProblem}
	{\bf }	
	\begin{homeworkSection}{Solution}
		
	\end{homeworkSection}
\end{homeworkProblem}
%=============================Problemi==========================%	
\newpage
\begin{homeworkProblem}
	{\bf  }	
	\begin{enumerate}[(a)]
		\item 
			\begin{homeworkSection}{Solution}
		
			\end{homeworkSection}
	\end{enumerate}
\end{homeworkProblem}
%=============================Problemi=========================%
\newpage
\begin{homeworkProblem}
	
	\begin{homeworkSection}{Solution}
		
	\end{homeworkSection}
\end{homeworkProblem}

\end{comment}%=========================================================

















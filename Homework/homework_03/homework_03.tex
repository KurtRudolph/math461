\documentclass{article}
\usepackage{hw_style}
\usepackage{enumerate}
\usepackage{graphicx}
\usepackage{verbatim}

% Homework Specific Information
\newcommand{\hmwkTitle}{Homework \#3}
\newcommand{\hmwkDueDate}{Tuesday, June 19th at 9:00AM}
\newcommand{\hmwkAuthorName}{Kurt Rudolph}%Name:
\newcommand{\hmwkNetID}{rudolph9}%your netid
\newcommand{\hmwkNotes}{}%I worked with...

\newcommand{\hmwkSubTitle}{}
\newcommand{\hmwkClass}{Math 461}
\newcommand{\hmwkClassTime}{}
\newcommand{\hmwkClassInstructor}{Kenneth B. Stolarsky}

\begin{document}
\begin{spacing}{1.1}
\maketitle
\newpage
%=============================Problem1==========================%	
\begin{homeworkProblem}
  {\bf Chapter 2, Theoretical Exercise 11} 

  If $P( E) = .9$ and $P( F) = .8$, show that $P( EF) \ge .7$. In general, prove Bonferroni’s inequality, namely,

  \[P( EF) \ge P( E) + P( F) - 1\]
  \begin{homeworkSection}{Bonferroni Inequality}
    Let $P( E_i)$ be the probability that $E_i$ is true, and $P\left( \bigcup\limits_{i = 1}^{n}{ E_i}\right)$ be the probability 
    that at least one of $E_1, E_2, \dots, E_n$ is true.  The the \emph{Bonferroni Inequality}, also knowon as \emph{Boole's Inequality} states
    \[P\left( \bigcup\limits_{i = 1}^{n}{ E_i}\right) \le \sum\limits_{i = 1}^{n}{ P( E_i)\]
  \end{homeworkSection}	

	\begin{homeworkSection}{Solution}
    
    Given from Proposition 4.2
      \[P( E \cup F) = P( E) + P( F) - P( EF)\]
    Since $P( E \cup F) \le 1$ it follows that 
      \[1 \ge P( E) + P( F) - P( EF)\]
    
		
	\end{homeworkSection}
\end{homeworkProblem}



\end{spacing}
\end{document}





\begin{comment}%==========================================================
%=============================Problemi==========================%	
\begin{homeworkProblem}
	
	\begin{homeworkSection}{Solution}
		
	\end{homeworkSection}
\end{homeworkProblem}
%=============================Problemi==========================%	
\begin{homeworkProblem}
	
	\begin{enumerate}[(a)]
		\item 
			\begin{homeworkSection}{Solution}
		
			\end{homeworkSection}
	\end{enumerate}
\end{homeworkProblem}
%=============================Problemi==========================%	
\begin{homeworkProblem}
	{\bf }	
	\begin{homeworkSection}{Solution}
		
	\end{homeworkSection}
\end{homeworkProblem}
%=============================Problemi==========================%	
\newpage
\begin{homeworkProblem}
	{\bf  }	
	\begin{enumerate}[(a)]
		\item 
			\begin{homeworkSection}{Solution}
		
			\end{homeworkSection}
	\end{enumerate}
\end{homeworkProblem}
%=============================Problemi=========================%
\newpage
\begin{homeworkProblem}
	
	\begin{homeworkSection}{Solution}
		
	\end{homeworkSection}
\end{homeworkProblem}

\end{comment}%=========================================================

















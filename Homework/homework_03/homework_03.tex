\documentclass{article}
\usepackage{hw_style}
\usepackage{enumerate}
\usepackage{graphicx}
\usepackage{verbatim}

% Homework Specific Information
\newcommand{\hmwkTitle}{Homework \#3}
\newcommand{\hmwkDueDate}{Tuesday, June 19th at 9:00AM}
\newcommand{\hmwkAuthorName}{Kurt Rudolph}%Name:
\newcommand{\hmwkNetID}{rudolph9}%your netid
\newcommand{\hmwkNotes}{}%I worked with...

\newcommand{\hmwkSubTitle}{}
\newcommand{\hmwkClass}{Math 461}
\newcommand{\hmwkClassTime}{}
\newcommand{\hmwkClassInstructor}{Kenneth B. Stolarsky}

\begin{document}
\begin{spacing}{1.1}
\maketitle
%=============================pg55 problem11==========================%	
\newpage
\begin{homeworkProblem}
  {\bf Chapter 2, Theoretical Exercise 11} 

  If $P( E) = .9$ and $P( F) = .8$, show that $P( EF) \ge .7$. In general, prove Bonferroni’s inequality, namely,

  \[P( EF) \ge P( E) + P( F) - 1\]

  \begin{homeworkSection}{Bonferroni Inequality}
    Let $P( E_i)$ be the probability that $E_i$ is true, and $P\left( \bigcup\limits_{i = 1}^{n}{ E_i}\right)$ be the probability 
    that at least one of $E_1, E_2, \dots, E_n$ is true.  The the \emph{Bonferroni Inequality}, also knowon as \emph{Boole's Inequality} states

    \[P\left( \bigcup\limits_{i = 1}^{n}{ E_i}\right) \le \sum\limits_{i = 1}^{n}{ P( E_i)}\]
  \end{homeworkSection}	

  \begin{homeworkSection}{Inclusion-Exclusion Principle}
    Given from Proposition 4.2
      \[P( E \cup F) = P( E) + P( F) - P( EF)\]
    The more general form
      \[\left|\bigcup\limits_{i = 1}^{n}{ E_i}\right| \le \sum\limits_{i = 1}^{n}{ \left|E_i\right|}\]
  \end{homeworkSection}

	\begin{homeworkSection}{Solution}
    Since $P( E \cup F) \le 1$ it follows that 
      \[1 \ge P( E) + P( F) - P( EF)\]
      \[P( EF) \ge P( E) + P( F) - 1\]
    This follows Bonferroni Inequality and show that
      \[P( EF) \ge 0.9 + 0.8 - 1 = 0.7\]
    
	\end{homeworkSection}
\end{homeworkProblem}

%=============================pg55 problem16==========================%	
\newpage
\begin{homeworkProblem}
  {\bf Chapter 2, Theoretical Exercise 16}
  
  Use induction to generalize Bonferroni’s inequality to $n$ events. That is, show that
    \[P( E_1 E_2 \cdots E_n) \ge P( E_1) + \cdots + P( E_n) - (n - 1)\]

  \begin{homeworkSection}{Bonferroni Inequality}
    Let $P( E_i)$ be the probability that $E_i$ is true, and $P\left( \bigcup\limits_{i = 1}^{n}{ E_i}\right)$ be the probability 
    that at least one of $E_1, E_2, \dots, E_n$ is true.  The the \emph{Bonferroni Inequality}, also knowon as \emph{Boole's Inequality} states

    \[P\left( \bigcup\limits_{i = 1}^{n}{ E_i}\right) \le \sum\limits_{i = 1}^{n}{ P( E_i)}\]
  \end{homeworkSection}	

  \begin{homeworkSection}{Inclusion-Exclusion Principle}
    Given from Proposition 4.2
      \[P( E \cup F) = P( E) + P( F) - P( EF)\]
    The more general form
      \[\left|\bigcup\limits_{i = 1}^{n}{ E_i}\right| \le \sum\limits_{i = 1}^{n}{ \left|E_i\right|}\]
  \end{homeworkSection}

	\begin{homeworkSection}{Solution}
    As shown in {\bf Chapter 2, Theoretical Exercise 11}, $P( EF) \ge P( E) + P( F) - 1$. 
    Using this inequality for two sets as our base case we must now consider the case where
    we have $n + 1$ sets.  A similar proof follows
      \[P( E_1 E_2 E_3 \cdots E_{n + 1}) \ge P( E_1) + P(E_2 E_3 \cdots E_{n + 1}) - 1\]
      \[P( E_1 E_2 E_3 \cdots E_{n + 1}) \ge P( E_1) + P( E_2) + P(E_3 E_4 \cdots E_{n + 1}) - 2\]
      \[P( E_1 E_2 E_3 \cdots E_{n + 1}) \ge \cdots\]
      \[P( E_1 E_2 E_3 \cdots E_{n + 1}) \ge P( E_1) + P( E_2) + \cdots P( E_{n + 1}) - (n)\]

    Since our basecase holds for $n = 2$, by inductive proof therefore 
      \[P( E_1 E_2 E_3 \cdots E_{n + 1}) \ge P( E_1) + P( E_2) + \cdots P( E_{n}) - (n - 1)\]

	\end{homeworkSection}
\end{homeworkProblem}

%=============================pg102 problem2==========================%	
\begin{homeworkProblem}
  {\bf Chapter 3, Problem 2	}
  
  If two fair dice are rolled, what is the conditional probability that the first one lands 
  on 6 given that the sum of the dice is $i$? Compute for all values of $i$ between 2 and 12.

	\begin{homeworkSection}{Solution}
    
		
	\end{homeworkSection}
\end{homeworkProblem}

\end{spacing}
\end{document}





\begin{comment}%==========================================================
%=============================Problemi==========================%	
\begin{homeworkProblem}
	
	\begin{homeworkSection}{Solution}
		
	\end{homeworkSection}
\end{homeworkProblem}
%=============================Problemi==========================%	
\begin{homeworkProblem}
	
	\begin{enumerate}[(a)]
		\item 
			\begin{homeworkSection}{Solution}
		
			\end{homeworkSection}
	\end{enumerate}
\end{homeworkProblem}
%=============================Problemi==========================%	
\begin{homeworkProblem}
	{\bf }	
	\begin{homeworkSection}{Solution}
		
	\end{homeworkSection}
\end{homeworkProblem}
%=============================Problemi==========================%	
\newpage
\begin{homeworkProblem}
	{\bf  }	
	\begin{enumerate}[(a)]
		\item 
			\begin{homeworkSection}{Solution}
		
			\end{homeworkSection}
	\end{enumerate}
\end{homeworkProblem}
%=============================Problemi=========================%
\newpage
\begin{homeworkProblem}
	
	\begin{homeworkSection}{Solution}
		
	\end{homeworkSection}
\end{homeworkProblem}

\end{comment}%=========================================================

















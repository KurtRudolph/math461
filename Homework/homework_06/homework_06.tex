\documentclass{article}
\usepackage{hw_style}
\usepackage{enumerate}
\usepackage{graphicx}
\usepackage{verbatim}

% Homework Specific Information
\newcommand{\hmwkTitle}{Homework \#6}
\newcommand{\hmwkDueDate}{Friday, June 29th, 9:00AM}
\newcommand{\hmwkAuthorName}{Kurt Rudolph}%Name:
\newcommand{\hmwkNetID}{rudolph9}%your netid
\newcommand{\hmwkNotes}{}%I worked with...

\newcommand{\hmwkSubTitle}{}
\newcommand{\hmwkClass}{Math 461}
\newcommand{\hmwkClassTime}{}
\newcommand{\hmwkClassInstructor}{Kenneth B. Stolarsky}

\begin{document}
\begin{spacing}{1.1}
\maketitle

%=============================p.180 #4==========================% 
\newpage
\begin{homeworkProblem}
  {\bf Chapter 4, Theoretical Exercise 4}
  For a nonnegative integer-valued random variable $N$, show that
  \[E[ N] = \sum\limits_{i = 1}^{\infty}{ P\{ N \le i\}}\]

  \emph{Hint:} $\sum\limits_{i = 1}^{\infty}{ P\{ N \le i\}} 
  = \sum\limits_{i = 1}^{ \infty}{ \sum\limits_{k = i}^{\infty}{P\{ N = k\}}}$.
  Now interchange the order of summation.
  \begin{homeworkSection}{Defintions}
    \begin{itemize}
      \item $F( x) = P\{ X \le x\}$ is the \emph{distribution function} of $X$
      \item $p( x) = P\{ X = x\}$ is the \emph{probability mass function} of $X$
      \item $E[ X] = \sum\limits_{x:p( x) > 0}{ x p( x)}$ is the 
      \emph{expected value} of $X$
    \end{itemize}
  \end{homeworkSection}
  \begin{homeworkSection}{Solution}
    Let $X = N$ be our random variable

    Since 
    \[E[ X] = \sum\limits_{x:p( x) > 0}{ x p( x)}\]
    \[p( x) = P\{ X = x\}\]
    \[\sum\limits_{i = 1}^{\infty}{ P\{ X \le i\}} 
    = \sum\limits_{i = 1}^{ \infty}{ \sum\limits_{k = i}^{\infty}{P\{ X = k\}}}\]
    We find 
    \[\sum\limits_{i = 1}^{\infty}{ P\{ X \le i\}} 
    = \sum\limits_{i = 1}^{ \infty}{ \sum\limits_{k = i}^{\infty}{P\{ X = k\}}}
    = \sum\limits_{i = 1}^{ \infty}{ \sum\limits_{k = i}^{\infty}{ p( k)}}\]
    By out outer summation, $p( x)$ will be summed $x$ times, therefore
    \[\sum\limits_{i = 1}^{\infty}{ P\{ X \le i\}} 
    = \sum\limits_{i = 1}^{ \infty}{ \sum\limits_{k = i}^{\infty}{P\{ X = k\}}}
    = \sum\limits_{i = 1}^{ \infty}{ \sum\limits_{k = i}^{\infty}{ p( k)}}
    = \sum\limits_{k:p( k) > 0}{ k p( k)} = E[ X]\]

  \end{homeworkSection}
\end{homeworkProblem}
%============================p.180 #10===========================% 
\newpage
\begin{homeworkProblem}
  {\bf Chapter 4, Theoretical Exercise 10}\\
  Let $X$ be a binomial random variable with parameters $n$ and $p$. Show that 
  \[E\left[\frac{ 1}{ X + 1}\right] = \frac{ 1 - (1 - p)^{n + 1}}{ (n + 1)p}\]
  \begin{homeworkSection}{Defintions}
    \begin{itemize}
      \item $E[ X] = \sum\limits_{x:p( x) > 0}{ x p( x)}$ is the 
      \emph{expected value} of $X$
      \item $p( i) = {n \choose i} p^i (1 - p)^{n - i}, i = 0, \dots, n$ is the
      probability mass function of a binomial random variable with perameters $n$
      \item $E[ X] = n p$ is the \emph{expected value} of a \emph{binomial}
      where $X \sim B( n, p)$
    \end{itemize}
  \end{homeworkSection}
  \begin{homeworkSection}{Solution}
    Following the definition of expectation we find
    \[E \left[\frac{ 1}{ X + 1}\right] 
    = \sum\limits_{x:p(x) > 0}{ \left(\frac{ 1}{ x + 1}\right) p( x)}\]
    Since $X \sim B( n, p)$ we find
    \[\sum\limits_{x:p(x) > 0}{ \left(\frac{ 1}{ x + 1}\right) p( x)} 
    = \sum\limits_{i = 0}^{n}{ \left(\frac{ 1}{ i + 1}\right) {n \choose i} p^i (1 - p)^{n - i}}\]
    We find from ${n \choose k} = \frac{ n}{ k} {n - 1 \choose k - 1}$
    that ${n + 1\choose k + 1} = \frac{ n + 1}{ k + 1} {n \choose k}$
    hence
    
    \[\sum\limits_{i = 0}^{n}{ \left(\frac{ 1}{ i + 1}\right) {n \choose i} p^i (1 - p)^{n - i}}
    = \sum\limits_{i = 0}^{n}{ \left(\frac{ 1}{ n + 1}\right) \left(\frac{ n + 1}{ i + 1}\right) {n \choose i} p^i (1 - p)^{n - i}}\]
    \[= \left(\frac{ 1}{ n + 1}\right) \sum\limits_{i = 0}^{n}{ {n + 1 \choose i + 1} p^i (1 - p)^{n - i}}\]
    \[= \left(\frac{ 1}{ p (n + 1)}\right) \sum\limits_{i = 0}^{n}{ {n + 1 \choose i + 1} p^{i + 1} (1 - p)^{n - i}}\]
    \[= \frac{ 1}{ p (n + 1)} (1 - (1 - p)^{n + 1})\]
    \[= \frac{ 1 - (1 - p)^{n + 1}}{ (n + 1)p}\]
  \end{homeworkSection}
\end{homeworkProblem}

%=============================p.180 16==========================% 
\newpage
\begin{homeworkProblem}
  {\bf Chapter 4, Theoretical Exercise 16}\\
  Let $X$ be a \emph{Poisson} random variable with parameter $\lambda$. 
  Show that $P\{ X = i\}$ increases monotonically and then decreases 
  monotonically as $i$ increases, reaching its maximum when $i$ is the 
  largest integer not exceeding $\lambda$.

  \emph{Hint:} Consider $\frac{ P\{ X = i\}}{ P\{ X = i - 1\}}$.
  \begin{homeworkSection}{Defintions}
    \begin{itemize}
      \item $p( i) = \frac{ e^{-\lambda} \lambda^i}{ i!}$ where 
        $i \ge 0$ is a \emph{Poisson} random variable with peramter $\lambda$ 
      \item $E[ X] = Var( X) = \lambda$ for a \emph{Poisson} random variable.
    \end{itemize}
  \end{homeworkSection}
 \begin{homeworkSection}{Solution}
    \[\frac{ P\{ X = i\}}{ P\{ X = i - 1\}} 
    = \frac{ \left(\frac{ e^{-\lambda} \lambda^i}{ i!}\right)}
          { \left(\frac{ e^{-\lambda} \lambda^{i - 1}}{ (i - 1)!}\right)}
    = \frac{ \lambda}{ i}\]
    hence when $i < \lambda$ $p( i)$ is increasing, when $ i < \lambda$ $p( i)$ is decreasing.
    Therefore, since $p( i)$ is a \emph{Poisson} distrobution where $E( x) = Var( x) = \lambda$,
    $p( i)$ is monotonically increasing and decreasing.
  \end{homeworkSection}
\end{homeworkProblem}

%=============================p.180 17b==========================% 
\newpage
\begin{homeworkProblem}
  {\bf Chapter 4, Theoretical Exercise 17b}\\
  Let $X$ be a Poisson random variable with parameter $\lambda$.
  Verify the formula
  \[P[ X \text{ is even}] = \frac{ 1}{ 2} \left[1 + e^{-2 \lambda}\right]\]
  directly by making use of the expansion of $e^{-\lambda} + e^{\lambda}$.
 \begin{homeworkSection}{Solution}
    Let $E$ be the event our \emph{Poisson} random variable is even.\\
    Since
    \[P( X) = \sum\limits_{i = 0}^{\infty}{ \frac{ e^{-\lambda} \lambda^{i}}{ i!}}\]
    it follows that
    \[P( E) = \sum\limits_{i = 0}^{\infty}{ \frac{e^{-\lambda} \lambda^{2 i}}{ (2 i)!}}\]
    making use of the expansion of $e^{-\lambda} + e^{\lambda}$ we see that 
    \[\cosh( \lambda) = \frac{ e^{-\lambda} + e^{\lambda}}{ 2}\]
    hence
    \[P( E) = \sum\limits_{i = 0}^{\infty}{ \frac{e^{-\lambda} \lambda^{2 i}}{ (2 i)!}}
    = e^{-\lambda} \sum\limits_{i = 0}^{\infty}{ \frac{ \lambda^{2 i}}{ (2 i)!}}
    = e^{-\lambda} \left(\frac{ e^{-\lambda} + e^{\lambda}}{ 2}\right) = \frac{ 1 + e^{-2 \lambda}}{ 2}\]
  \end{homeworkSection}
\end{homeworkProblem}

%=============================p.181 19==========================% 
\newpage
\begin{homeworkProblem}
  {\bf Chapter 4, Theoretical Exercise 19}\\
  Show that $X$ is a Poisson random variable with
  parameter $\lambda$, then
  \[E[ X^n] = \lambda E[ (X + 1)^{n - 1}]\]

  Now use this result to compute $E[ X^3]$.
  
 \begin{homeworkSection}{Solution}
    \[E[ X^3] = \lambda E[ (X + 1)^2]\] 
    \[= \lambda E[ X^2 + 2X + 1]\] 
    \[= \lambda^2 (E[ X + 1] + E[ 2x] + E[ 1])\]
    \[= \lambda^2( \lambda + 2) = \lambda^3 + 2 \lambda^2\]
    
  \end{homeworkSection}
\end{homeworkProblem}

%=============================p.177 51 ==========================% 
\newpage
\begin{homeworkProblem}
  {\bf Chapter 4, Exercise 51}\\
  The expected number of typographical errors on a page of a certain 
  magazine is $0.2$. What is the probability that the next page you 
  read contains {\bf (a)} 0 and {\bf (b)} 2 or more typographical 
  errors? Explain your reasoning!
 
  \begin{enumerate}[(a)] 
    \item 0
      \begin{homeworkSection}{Solution}
        \[1 - P{ X \ge 1} = 1 - .2 = .8\]
        Were calculating the probability there are exactly zero errors.
      \end{homeworkSection}
    \item 2
      \begin{homeworkSection}{Solution}
      \[ P\{ X \ge 2\} = (P\{ X \ge 1\}) (P\{ X \ge 1\}) = (.2)(.2) = .04\]
      Treating the errors as independent we find that it is simply $P\{ X \ge 1\}^2$
      \end{homeworkSection}
  \end{enumerate}
\end{homeworkProblem}

%=============================p.177 57==========================% 
\newpage
\begin{homeworkProblem}
  {\bf Chapter 4, Exercise 57}\\
  Suppose that the number of accidents occurring on a highway each day 
  is a Poisson random variable with parameter $\lambda = 3$.
 
  \begin{homeworkSection}{Defintions}
    \begin{itemize}
      \item $p( i) = \frac{ e^{-\lambda} \lambda^i}{ i!}$ where 
        $i \ge 0$ is a \emph{Poisson} random variable with peramter $\lambda$ 
      \item $E[ X] = Var( X) = \lambda$ for a \emph{Poisson} random variable.
    \end{itemize}
  \end{homeworkSection}
  \begin{enumerate}[(a)]
    \item Find the probability that 3 or more accidents occur today.
      \begin{homeworkSection}{Solution}
        \[P\{ X \ge 3 \} = 1 - P\{ X = 0\} = P\{ X = 1\} - P\{ X = 2\}\]
        \[= 1 - e^{-\lambda} - \lambda e^{-\lambda} - \frac{ e^{-\lambda} \lambda^2}{ 2} = .5768\]
      \end{homeworkSection}
    \item Repeat part (a) under the assumption that at least 1 accident 
      occurs today.
      \begin{homeworkSection}{Solution}
        \[P\{ X \ge 3 | X \ge 1\} = \frac{ P\{ X \ge 3\}}{ P\{ X \ge 1}\}\]
        Since
        \[P\{X \ge 1\} = 1 - e^{-\lambda} = .95\]
        \[P\{ X \ge 3 | X \ge 1\} = \frac{ .5768}{ .95} = .607\]
      \end{homeworkSection}
  \end{enumerate}
\end{homeworkProblem}

%=============================p.177 58ab==========================% 
\newpage
\begin{homeworkProblem}
  {\bf Chapter 4, Exercise 58ab}\\
  Compare the Poisson approximation with the correct binomial probability 
  for the following cases:
 
  \begin{homeworkSection}{Defintions}
    \begin{itemize}
      \item $E[ X] = \sum\limits_{x:p( x) > 0}{ x p( x)}$ is the 
      \emph{expected value} of $X$
      \item $p( i) = {n \choose i} p^i (1 - p)^{n - i}, i = 0, \dots, n$ is the
      probability mass function of a binomial random variable with perameters $n$
      \item $E[ X] = n p$ is the \emph{expected value} of a \emph{binomial}
      where $X \sim B( n, p)$
      \item $p( i) = \frac{ e^{-\lambda} \lambda^i}{ i!}$ where 
        $i \ge 0$ is a \emph{Poisson} random variable with peramter $\lambda$ 
      \item $E[ X] = Var( X) = \lambda$ for a \emph{Poisson} random variable.
    \end{itemize}
  \end{homeworkSection}
  I do not believe I understand what to use as $\lambda$
  \begin{enumerate}[(a)]
    \item $P[ X = 2]$ when $n = 8, p = 0.1$
      \begin{homeworkSection}{Solution}
        Possion, let $\lambda = np = .8$
         \[p(2) = \frac{ e^{-\lambda} \lambda^2}{ 2} = .143785\]
        Binomial
        \[P(2) = {8 \choose 2} p^2 (1 - p)^6 = .148803\]
        Comparison
        \[\frac{ |.148803 -.143785|}{ .148803} = .03 = 3\%\]
      \end{homeworkSection}
    \item $P[ X = 9]$ when $n = 10, p = 0.95$
      \begin{homeworkSection}{Solution}
        Possion, let $\lambda = np = 9.5$
         \[p(9) = \frac{ e^{-\lambda} \lambda^9}{ 9!} = .130003\]
        Binomial
        \[P(9) = {10 \choose 9} p^9 (1 - p)^1 = .3151\]
        Comparison
        \[\frac{|.3151 -.130003|}{.3151} = .587 = 58\%\]
      \end{homeworkSection}
  \end{enumerate}
\end{homeworkProblem}

%=============================p.177 59==========================% 
\newpage
\begin{homeworkProblem}
  {\bf Chapter 4, Exercise 59}\\
  If you buy a lottery ticket in 50 lotteries, in each of which your 
  chance of winning a prize is $\frac{ 1}{ 100}$ , what
  is the (approximate) probability that you will win
  a prize.
  \begin{homeworkSection}{Defintions}
    \begin{itemize}
      \item $p( i) = \frac{ e^{-\lambda} \lambda^i}{ i!}$ where 
        $i \ge 0$ is a \emph{Poisson} random variable with peramter $\lambda$ 
      \item $E[ X] = Var( X) = \lambda$ for a \emph{Poisson} random variable.
    \end{itemize}
  \end{homeworkSection}
  \begin{enumerate}[(a)]
    \item at least once?
      \begin{homeworkSection}{Solution}
        \[p(1) = .3935\]
      \end{homeworkSection}
    \item exactly once?
      \begin{homeworkSection}{Solution}
        \[p( 1) - p( 2) = .3033\]
      \end{homeworkSection}
    \item at least twice?
      \begin{homeworkSection}{Solution}
        \[p(2) = .0902\]
      \end{homeworkSection}
  \end{enumerate}
\end{homeworkProblem}
\end{spacing}
\end{document}

\begin{comment}%==========================================================

%=============================Problemi==========================% 
\newpage
\begin{homeworkProblem}
  
 \begin{homeworkSection}{Solution}
    
  \end{homeworkSection}
\end{homeworkProblem}
%=============================Problemi==========================% 
\begin{homeworkProblem}
  
  \begin{enumerate}[(a)]
    \item 
      \begin{homeworkSection}{Solution}
    
      \end{homeworkSection}
  \end{enumerate}
\end{homeworkProblem}
Homework #6 is due Friday, 6/29: Page 180: #4, 10, 16, 17 (b), 19; Page 177: 51, 57, 58 (a,b), 59
\end{comment}

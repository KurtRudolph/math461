\documentclass{article}
\usepackage{hw_style}
\usepackage{enumerate}
\usepackage{graphicx}
\usepackage{verbatim}

% Homework Specific Information
\newcommand{\hmwkTitle}{Homework \#8}
\newcommand{\hmwkDueDate}{ Wednesday July 11th, 2012, 9:00AM}
\newcommand{\hmwkAuthorName}{Kurt Rudolph}%Name:
\newcommand{\hmwkNetID}{rudolph9}%your netid
\newcommand{\hmwkNotes}{}%I worked with...

\newcommand{\hmwkSubTitle}{}
\newcommand{\hmwkClass}{Math 461}
\newcommand{\hmwkClassTime}{}
\newcommand{\hmwkClassInstructor}{Kenneth B. Stolarsky}

\begin{document}
\begin{spacing}{1.1}
\maketitle

%=============================p.225 #19==========================% 
\newpage
\begin{homeworkProblem}
  {\bf Chapter 5, Problem 19}\\
  Let $X$ be a normal random variable with mean 12 and variance 4. 
  Find the value of $c$ such that $P\{X > c\} = 0.1$.
  \begin{homeworkSection}{Solution}
    
  \end{homeworkSection}
\end{homeworkProblem}

%=============================p.225 #20==========================% 
\newpage
\begin{homeworkProblem}
  {\bf Chapter 5, Problem 20c}\\
  If 65 percent of the population of a large community is in favor of a 
  proposed rise in school taxes, approximate the probability that a random 
  sample of 100 people will contain fewer than 75 in favor?
  \begin{homeworkSection}{Solution}
    
  \end{homeworkSection}
\end{homeworkProblem}

%=============================p.225 #21==========================% 
\newpage
\begin{homeworkProblem}
  {\bf Chapter 5, Problem 21}\\
  Suppose that the height, in inches, of a 25-year-old man is a normal random 
  variable with parameters $\mu = 71, \sigma^2 = 6.25$.  
  What percentage of 25-year-old men are over 6 feet, 2 inches tall? 
  What percentage of men in the 6-footer club are over 6 feet, 5 inches?
  \begin{homeworkSection}{Solution}
    
  \end{homeworkSection}
\end{homeworkProblem}

%=============================p.227 #39==========================% 
\newpage
\begin{homeworkProblem}
  {\bf Chapter 5, Problem 39}\\
  If $X$ is an exponential random variable with parameter $\lambda = 1$,
  compute the probability density function of the random variable $Y$ 
  defined by $Y = \log X$.
  \begin{homeworkSection}{Solution}
    
  \end{homeworkSection}
\end{homeworkProblem}

%=============================p.227 #40==========================% 
\newpage
\begin{homeworkProblem}
  {\bf Chapter 5, Problem 40}\\
  If $X$ is uniformly distributed over $(0, 1)$, find the density 
  function of $Y = e^X$ .
  \begin{homeworkSection}{Solution}
    
  \end{homeworkSection}
\end{homeworkProblem}

%=============================p.228 #9==========================% 
\newpage
\begin{homeworkProblem}
  {\bf Chapter 5, Theoretical Exercise 9}\\
  Show that $Z$ is a standard normal random variable,
  then, for $x > 0$,
  \begin{enumerate}[(a)]
    \item $P\{ Z > x\} = P\{ Z < −x\}$
      \begin{homeworkSection}{Solution}
        
      \end{homeworkSection}
    \item $P\{ |Z| > x\} = 2 P\{ Z > x\}$
      \begin{homeworkSection}{Solution}
        
      \end{homeworkSection}
    \item $P\{ |Z| < x \} = 2 P\{ Z < x \} − 1$
      \begin{homeworkSection}{Solution}
        
      \end{homeworkSection}
    \end{enumerate}
\end{homeworkProblem}

%=============================p.228 #11a==========================% 
\newpage
\begin{homeworkProblem}
  {\bf Chapter 5, Theoretical Exercise 11a}\\
  Let $Z$ be a standard normal random variable $Z$, and let 
  $g$ be a differentiable function with derivative $g'$.
  Find $E[ Z^4]$.
  \begin{homeworkSection}{Solution}
    
  \end{homeworkSection}
\end{homeworkProblem}

%=============================p.228 #12==========================% 
\newpage
\begin{homeworkProblem}
  {\bf Chapter 5, Theoretical Exercise 12}\\
  Use the identity of Theoretical Exercise 5 to derive $E[ X^2]$ when $X$ 
  is an exponential random variable with parameter $\lambda$.
  \begin{homeworkSection}{Solution}
    
  \end{homeworkSection}
\end{homeworkProblem}

%=============================p.228 #12==========================% 
\newpage
\begin{homeworkProblem}
  {\bf Chapter 5, Theoretical Exercise 27}\\
  If $X$ is uniformly distributed over $(a, b)$, what random variable, 
  having a linear relation with $X$, is uniformly distributed over $(0, 1)$?
  \begin{homeworkSection}{Solution}
    
  \end{homeworkSection}
\end{homeworkProblem}
  
\end{spacing}
\end{document}

\begin{comment}%==========================================================
* p.225-227
  * 19
  * 20c
  * 21
  * 39
  * 40
* p.228
  * 9
  * 11a
  * 12
  * 27

%=============================Problemi==========================% 
\newpage
\begin{homeworkProblem}
  
  \begin{homeworkSection}{Solution}
    
  \end{homeworkSection}
\end{homeworkProblem}
%=============================Problemi==========================% 
\begin{homeworkProblem}
  
  \begin{enumerate}[(a)]
    \item 
      \begin{homeworkSection}{Solution}
    
      \end{homeworkSection}
  \end{enumerate}
\end{homeworkProblem}

\end{comment}

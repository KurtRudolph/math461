\documentclass{article}
\usepackage{hw_style}
\usepackage{enumerate}
\usepackage{graphicx}
\usepackage{verbatim}
\usepackage{amsmath}

% Homework Specific Information
\newcommand{\hmwkTitle}{Homework \#8}
\newcommand{\hmwkDueDate}{ Wednesday July 11th, 2012, 9:00AM}
\newcommand{\hmwkAuthorName}{Kurt Rudolph}%Name:
\newcommand{\hmwkNetID}{rudolph9}%your netid
\newcommand{\hmwkNotes}{}%I worked with...

\newcommand{\hmwkSubTitle}{}
\newcommand{\hmwkClass}{Math 461}
\newcommand{\hmwkClassTime}{}
\newcommand{\hmwkClassInstructor}{Kenneth B. Stolarsky}

\begin{document}
\begin{spacing}{1.1}
\maketitle

%=============================p.225 #19==========================% 
\newpage
\begin{homeworkProblem}
  {\bf Chapter 5, Problem 19}\\
  Let $X$ be a normal random variable with mean 12 and variance 4. 
  Find the value of $c$ such that $P\{X > c\} = 0.1$.
  \begin{homeworkSection}{Solution}
    \begin{align*}
      P\{X > c\} &= P\{\frac{ X - \mu}{ \sigma} > \frac{ c - \mu}{ \sigma}\}\\
      &= 1 - P\{\frac{ X - \mu}{ \sigma} < \frac{ c - \mu}{ \sigma}\}\\
      &= 1 - \Phi(\frac{ c - \mu}{ \sigma}) = 0.1\\
    \end{align*}
    Hence
    \[\Phi(\frac{ c - \mu}{ \sigma}) = 0.9\]
    Solving for $c$
    \[\frac{ c - \mu}{ \sigma} = \Phi^{-1}( 0.9)\]
    \begin{align*}
      c &= \sigma \Phi^{-1}(0.9) + \mu\\
        &= (2) (1.28) + (12) = 14.56
    \end{align*}
  \end{homeworkSection}
\end{homeworkProblem}

%=============================p.225 #20==========================% 
\newpage
\begin{homeworkProblem}
  {\bf Chapter 5, Problem 20c}\\
  If 65 percent of the population of a large community is in favor of a 
  proposed rise in school taxes, approximate the probability that a random 
  sample of 100 people will contain fewer than 75 in favor?
  \begin{homeworkSection}{Solution}
    Let $n = 100, p= 0.65$.  Using the normal approximation to a binomail we
    find
    \[\mu = np = 65\]
    \[\sigma^2 = np(1 - p) = 22.75\]
    \begin{align*}
      P\{X < 75\} &= P\{X < 74.5\}\\
                &= P\{ Z < \frac{ X - \mu}{ \sigma}\}\\
                &= P\{ Z < \frac{ 74.5 - 65}{ 4.76}\}\\
                &= \Phi( 1.99) \approx 0.97967
    \end{align*}
  \end{homeworkSection}
\end{homeworkProblem}

%=============================p.225 #21==========================% 
\newpage
\begin{homeworkProblem}
  {\bf Chapter 5, Problem 21}\\
  Suppose that the height, in inches, of a 25-year-old man is a normal random 
  variable with parameters $\mu = 71, \sigma^2 = 6.25$.  
  What percentage of 25-year-old men are over 6 feet, 2 inches tall? 
  What percentage of men in the 6-footer club are over 6 feet, 5 inches?
  \begin{homeworkSection}{Solution}
    \[\mu = 71\]
    \[\sigma^2 = 6.25\]
    \begin{align*}
      P\{ X > 74 \} &= P\{ Z > \frac{ X - \mu}{ \sigma}\}\\
                    &= 1 - P\{ Z < \frac{ X - \mu}{ \sigma}\}\\
                    &= 1 - P\{ Z < \frac{ 74 - 71}{ 2.5}\}\\
                    &= 1 - \Phi( 1.2) \approx 0.1151
    \end{align*}
    Now to find the percentage of men in the 6-footer club who are over 6 feet, 5 inches
    we look at the conditional probability
    \begin{align*}
      P\{ X > 77 | X > 72\} &= \frac{ P\{ X > 77\}}{ P\{ X > 72\}}\\
                            &= \frac{ 1 - P\{ Z < \frac{ 77 - \mu}{ \sigma}\}}{ 1 - P\{ Z < \frac{ 72 - \mu}{ \sigma}\}}\\
                            &= \frac{ 1 - \Phi( 2.4)}{ 1 - \Phi( 0.4)} \approx 0.0238
    \end{align*}
  \end{homeworkSection}
\end{homeworkProblem}

%=============================p.227 #39==========================% 
\newpage
\begin{homeworkProblem}
  {\bf Chapter 5, Problem 39}\\
  If $X$ is an exponential random variable with parameter $\lambda = 1$,
  compute the probability density function of the random variable $Y$ 
  defined by $Y = \log X$.
  \begin{homeworkSection}{Solution}
    \begin{align*}
      F_Y( a) &= P\{ Y \le a\} \\
              &= P\{ \log( X) \le a\} \\
              &= P\{ X \le e^a\} \\
              &= F_X( e^a)
    \end{align*}
    Since $X$ is an exponential random variable
    \begin{align*}
      F_X( a) &= 1 - e^{-\lambda a} \\
              &= 1 - e^{-a}
    \end{align*}
    Hence
    \[F_Y( a) = 1 - e^{-e^a}\]
    Therefore, the pdf of the random variable $Y$ is
    \begin{align*}
      f_Y( a) &= \frac{ d F_Y( a)}{ d a} \\
              &= \frac{ d}{ da}( 1 - e^{-e^a}) \\
              &= (-e^a) (e^{-e^a}) \\
              &= e^{a + e^a}
    \end{align*}
  \end{homeworkSection}
\end{homeworkProblem}

%=============================p.227 #40==========================% 
\newpage
\begin{homeworkProblem}
  {\bf Chapter 5, Problem 40}\\
  If $X$ is uniformly distributed over $(0, 1)$, find the density 
  function of $Y = e^X$ .
  \begin{homeworkSection}{Solution}
    \begin{align*}
      F_Y( a) &= P\{ Y \le a\}\\
              &= P\{ e^X \le a\}\\
              &= P\{ X \le \log( a)\}\\
              &= F_X( \log( a))
    \end{align*}
    Since $X$ is unformly distributed, it's \emph{cdf} is linear 
    \[F_X( a) = a\]
    \[F_Y( a_ = \log( a)\]
    Therefore the \emph{pdf} of $Y$ is
    \begin{align*}
      f_Y( a) &= \frac{ d F_Y( a)}{ d a}\\ 
              &= \frac{ d}{ da} ( \log( a))\\
              &= \frac{ 1}{ a} \text{ where } 1 \le a \le e
    \end{align*}
  \end{homeworkSection}
\end{homeworkProblem}

%=============================p.228 #9==========================% 
\newpage
\begin{homeworkProblem}
  {\bf Chapter 5, Theoretical Exercise 9}\\
  Show that $Z$ is a standard normal random variable,
  then, for $x > 0$,
  \begin{enumerate}[(a)]
    \item $P\{ Z > x\} = P\{ Z < −x\}$
      \begin{homeworkSection}{Solution}
          I don't know
        
      \end{homeworkSection}
    \item $P\{ |Z| > x\} = 2 P\{ Z > x\}$
      \begin{homeworkSection}{Solution}
       I don't know 
      \end{homeworkSection}
    \item $P\{ |Z| < x \} = 2 P\{ Z < x \} − 1$
      \begin{homeworkSection}{Solution}
        I don't know 
      \end{homeworkSection}
    \end{enumerate}
\end{homeworkProblem}

%=============================p.228 #11a==========================% 
\newpage
\begin{homeworkProblem}
  {\bf Chapter 5, Theoretical Exercise 11a}\\
  Let $Z$ be a standard normal random variable $Z$, and let 
  $g$ be a differentiable function with derivative $g'$.
  Show that $E[ g'(Z)] = E[Zg(Z)]$.
  \begin{homeworkSection}{Solution}
    Using the identity
      \[E[ g(x)] = \int\limits_{-\infty}^{\infty} g(x)f(x) dx \]
    We find
    \begin{align*}
      E[Zg(Z)]  &= \int\limits_{-\infty}^{\infty} g'(Z)f(x) dx\\
                &= g(Z)f(x) -  \int\limits_{-\infty}^{\infty} g(Z)f'(x) dx\\
                &= \int\limits_{-\infty}^{\infty} Zg(Z)f(x) dx
    \end{align*}
  \end{homeworkSection}
\end{homeworkProblem}

%=============================p.228 #12==========================% 
\newpage
\begin{homeworkProblem}
  {\bf Chapter 5, Theoretical Exercise 12}\\
  Use the identity of Theoretical Exercise 5 to derive $E[ X^2]$ when $X$ 
  is an exponential random variable with parameter $\lambda$.
  \begin{homeworkSection}{Solution}
    From Theoretical Exercise 5
      \[E[ X^n] = \int\limits_{0}^{\infty} n x^{n - 1 } P\{ X > x \} dx\]
    Since $X$ is an exponential random variable 
      \[P\{ X \le x\} = 1 - e^{-\lambda x}\]
    is the \emph{cdf}.  Plugging into the formula for expectation from 
    Theoretical Exercise 5, we find
    \begin{align*}
      E[ X^2] &= \int\limits_{0}^{\infty} 2 x e^{-\lambda x} dx\\
              &= 2 \int\limits_{0}^{\infty} x e^{-\lambda x} dx\\
              &= 2\left[ \left. x e^{-\lambda x} (-\lambda)^{-1} \right|_{0}^{\infty} + \lambda^{-1}\int\limits_0^{\infty} e^{-\lambda x} dx\right] \\
              &= \frac{ 2}{ \lambda^2}
    \end{align*}
    
  \end{homeworkSection}
\end{homeworkProblem}

%=============================p.228 #27==========================% 
\newpage
\begin{homeworkProblem}
  {\bf Chapter 5, Theoretical Exercise 27}\\
  If $X$ is uniformly distributed over $(a, b)$, what random variable, 
  having a linear relation with $X$, is uniformly distributed over $(0, 1)$?
  \begin{homeworkSection}{Solution}
    The binomial Random Variable, I'm not sure what this question is asking for. 
    
  \end{homeworkSection}
\end{homeworkProblem}
  
\end{spacing}
\end{document}

\begin{comment}%==========================================================
* p.225-227
  * 19
  * 20c
  * 21
  * 39
  * 40
* p.228
  * 9
  * 11a
  * 12
  * 27

%=============================Problemi==========================% 
\newpage
\begin{homeworkProblem}
  
  \begin{homeworkSection}{Solution}
    
  \end{homeworkSection}
\end{homeworkProblem}
%=============================Problemi==========================% 
\begin{homeworkProblem}
  
  \begin{enumerate}[(a)]
    \item 
      \begin{homeworkSection}{Solution}
    
      \end{homeworkSection}
  \end{enumerate}
\end{homeworkProblem}

\end{comment}

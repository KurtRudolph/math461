\documentclass{article}
\usepackage{hw_style}
\usepackage{enumerate}
\usepackage{graphicx}
\usepackage{verbatim}
\usepackage{amsmath}
\usepackage{mathtools}

% Homework Specific Information
\newcommand{\hmwkTitle}{Homework \#13}
\newcommand{\hmwkDueDate}{Monday, July 30, 2012}
\newcommand{\hmwkAuthorName}{Kurt Rudolph}%Name:
\newcommand{\hmwkNetID}{rudolph9}%your netid
\newcommand{\hmwkNotes}{}%I worked with...

\newcommand{\hmwkSubTitle}{}
\newcommand{\hmwkClass}{Math 461}
\newcommand{\hmwkClassTime}{}
\newcommand{\hmwkClassInstructor}{Kenneth B. Stolarsky}

\begin{document}
\begin{spacing}{1.1}
\maketitle

%=============================p.373 #3==========================% 
\newpage
\begin{homeworkProblem}
  {\bf Chapter 7, Problem 3}
  Gambles are independent, and each one results in
  the player being equally likely to win or lose 1 unit. 
  Let $W$ denote the net winnings of a gambler whose strategy 
  is to stop gambling immediately after his first win. Find
  \begin{enumerate}[(a)]
    \item $P\{ W > 0 \}$
      \begin{homeworkSection}{Solution}
        Given, the gambler stops immediately after his
        first win, we only need to look at one trial to determine $P\{ W > 0 \}$.
        Since $P\{ W\} = \frac{ 1}{ 2}$, therefore $P\{ W > 0\} = \frac{ 1}{ 2}$ and 
      \end{homeworkSection}
    \item $P\{ W < 0 \}$
      \begin{homeworkSection}{Solution}
        Since our gambler will gamble until he receives a win, where he will gain
        one unit, hence must receive at least two consecutive initial losses in
        oder for $W$ to be negative, therefore
        $P\{ W < 0 \} = \left(\frac{ 1}{ 2}\right)\left(\frac{ 1}{ 2}\right) = \frac{ 1}{ 4}$.
      \end{homeworkSection}
    \item $E[ W]$
      \begin{homeworkSection}{Solution}
        Let $W$ be the amount of \emph{winnings} and let $X$ be the number of trials until the
        first positive earning win is received.
        \begin{align*}
          W &= 1 - (X - 1)\\
          &= 2 - X\\
        \end{align*}
        Since the probability of receiving a positive earning win is $\frac{ 1}{ 2}$ and the
        random variable $X$ is a geometric random variable we find
        \begin{align*}
          E[ X] &= \frac{ 1}{ p}\\ 
          &= \frac{ 1}{ 1/2} \\
          &= 2
        \end{align*}
        Therefore
        \begin{align*}
          E[ W] &= E[ 2 - X]\\
          &= 2 - E[ X]\\
          &= 0
        \end{align*}
      \end{homeworkSection}
  \end{enumerate}
\end{homeworkProblem}

%=============================p.373 #4==========================% 
\newpage
\begin{homeworkProblem}
  {\bf Chapter 7, Problem 4}
  Given $X$ and $Y$ have joint density function
  \[
    f_{X, Y}( x, y) = \left\{ \begin{array}{l l}
      1/y &0 < x < y < 1\\
      0   &\text{otherwise}
    \end{array}\right.
  \]
  find
  \begin{enumerate}[(a)]
    \item $E[XY]$
      \begin{homeworkSection}{Solution}
        \begin{align*}
          E[ XY] &=  \int\limits_0^1 \int\limits_0^y xy\frac{ 1}{ y} dx dy\\
          E[ XY] &=  \int\limits_0^1 \int\limits_0^y x dx dy\\
          &= \int\limits_0^1 \frac{ y^2}{ 2} dy\\
          &= \frac{ 1}{ 6}\\
        \end{align*}
      \end{homeworkSection}
    \item $E[X]$
      \begin{homeworkSection}{Solution}
        \begin{align*}
          E[ X] &=  \int\limits_0^1 \int\limits_0^y x\frac{ 1}{ y} dx dy\\
          &= \int\limits_0^1 \frac{ y}{ 2} dy\\
          &= \frac{ 1}{ 4}\\
        \end{align*}
      \end{homeworkSection}
    \item $E[Y]$
      \begin{homeworkSection}{Solution}
        \begin{align*}
          E[ Y] &=  \int\limits_0^1 \int\limits_0^y y\frac{ 1}{ y} dx dy\\
          &=  \int\limits_0^1 \int\limits_0^y 1 dx dy\\
          &=  \int\limits_0^1 y dy\\
          &=  \frac{ 1}{ 2}\\
        \end{align*}
      \end{homeworkSection}
  \end{enumerate}
\end{homeworkProblem}

%=============================p.376 #40==========================% 
\newpage
\begin{homeworkProblem}
  {\bf Chapter 7, Problem 40}\\
  Given $X$ and $Y$ have joint density function
  \[
    f_{X, Y}( x, y) = \begin{array}{l l}
      \frac{ 1}{ y} e^{-(y + \frac{ x}{ y})} & x < 0 < y\\
    \end{array}
  \]
  \begin{homeworkSection}{Solution}
    \begin{align*}
      E[ X] &= \int\limits_{x = 0}^\infty x \int_{y = 0}^\infty f( x, y) dy dx\\
      &= \int\limits_{x = 0}^\infty \int_{y = 0}^\infty \frac{ x}{ y} e^{-(y + x / y)} dy dx\\
      &= \int\limits_{y = 0}^\infty \int_{x = 0}^\infty \frac{ x}{ y} e^{-(y + x / y)} dx dy\\
      \text{Let $ v = \frac{ x}{ y}$ where $dv = \frac{ dx}{ y}$}\\
      &= \int\limits_{y = 0}^\infty \int_{v = 0}^\infty v e^{-(y + v)} y dv dy\\
      &= \int\limits_{y = 0}^\infty \int_{v = 0}^\infty v e^{-y} e^{-v} y dv dy\\
      &= \int\limits_{y = 0}^\infty e^{-y} y dy\\
      &= 1
    \end{align*}
    \begin{align*}
      E[ Y] &= \int\limits_{y = 0}^\infty y \int_{x = 0}^\infty y f( x, y) dx dy\\
      &= \int\limits_{y = 0}^\infty \int_{x = 0}^\infty e^{-(y + x / y)} dx dy\\
      \text{Let $ v = \frac{ x}{ y}$ where $dv = \frac{ dx}{ y}$}\\
      &= \int\limits_{y = 0}^\infty \int_{v = 0}^\infty e^{-(y + v)} y dv dy\\
      &= \int\limits_{y = 0}^\infty \int_{v = 0}^\infty e^{-y} e^{-v} y dv dy\\
      &= \int\limits_{y = 0}^\infty  e^{-y} y dy\\
      &= 1
    \end{align*}
    \begin{align*}
      E[ XY] &= \int\limits_{y = 0}^\infty y \int_{x = 0}^\infty xy f( x, y) dx dy\\
      &= \int\limits_{y = 0}^\infty \int_{x = 0}^\infty xe^{-(y + x / y)} dx dy\\
      \text{Let $ v = \frac{ x}{ y}$ where $dv = \frac{ dx}{ y}$}\\
      &= \int\limits_{y = 0}^\infty \int_{v = 0}^\infty v e^{-(y + v)} y^2 dv dy\\
      &= \int\limits_{y = 0}^\infty \int_{v = 0}^\infty e^{-y} e^{-v} y^2 dv dy\\
      &= \int\limits_{y = 0}^\infty  e^{-y} y^2 dy\\
      &= 2
    \end{align*}
    Therefore, \[Cov(X, Y) = E[XY] - E[X]E[Y] = 2 - 1 = 1\]
  \end{homeworkSection}
\end{homeworkProblem}

%=============================p.376 #48==========================% 
\newpage
\begin{homeworkProblem}
  {\bf Chapter 7 Problem 48}\\
  A fair die is successively rolled. Let $X$ and $Y$ denote, 
  respectively, the number of rolls necessary to obtain a 
  6 and a 5. Find
  \begin{enumerate}[(a)]
    \item Find $E[X]$
      \begin{homeworkSection}{Solution}
        Since $p = \frac{ 1}{ 5}$ and $X$ is a gemetric random variable we find
        \[E[ X] = \frac{ 1}{ p} = 6\]
      \end{homeworkSection}
    \item Find $E[X|Y = 1]$
      \begin{homeworkSection}{Solution}
        The expectation of the number of roles to recieve a 6 given that first
        role is a 5, building off our result from (a), is as follows
        \[E[ X|Y = 1] = 1 + E[ X] = 1 + \frac{ 1}{ p} = 1 + 6 = 7\]
      \end{homeworkSection}
    \item Show $E[X|Y = 5]$
      \begin{homeworkSection}{Solution}
        The expectation of the number of roles to recieve a 6 given that the 5th role is a 5
        I don't know...
      \end{homeworkSection}
  \end{enumerate}
\end{homeworkProblem}

%=============================p.376 #50==========================% 
\newpage
\begin{homeworkProblem}
  {\bf Chapter 7 Problem 50}\\
  Given $X$ and $Y$ have joint density function
  \[
    f_( x, y) = \begin{array}{l l}
      \frac{ e^{-x/y} e^{-y}}{ y} & 0 < x,y < \infty\\
    \end{array}
  \]
  Find $E[ X^2|Y = y]$
  \begin{homeworkSection}{Solution}
    \begin{align*}
      E[ X^2|Y = y]
        &= \int\limits_0^\infty x^2 f(x | Y = y) dx\\
      \text{Solving for $f(x | Y = y)$}\\
      f(x | Y = y) 
        &= \frac{ f( x, y)}{ f( y)}\\
        \text{where}\\
        f( y) &= \int\limits_0^\infty f( x, y) dx\\
        &= \int\limits_0^\infty \frac{ e^{-x/y} e^{-y}}{ y} dx\\
        &= e^{-y}\\
        \text{hence}\\
      \frac{ f( x, y)}{ f( y)}
        &= \frac{ \frac{ e^{-x/y} e^{-y}}{ y}}{ e^{-y}}\\
        &= \frac{ e^{-x/y}}{ y}\\
    \end{align*}
    \begin{align*}
      \text{pluggin into our expectation}\\
      \int\limits_0^\infty x^2 f(x | Y = y) dx\\
      &= \int\limits_0^\infty x^2 \frac{ e^{-x/y}}{ y} dx\\
      &= \frac{ 1}{ y} \int\limits_0^\infty x^2 e^{-x/y} dx\\
      \text{integrating by parts}\\
      &= \frac{ 1}{ y}\left( \left.x^2(-y)e^{-x/y}\right|_0^\infty - \int\limits_0^\infty 2x e^{-x/y} dx\right)\\
      &= 2 \int\limits_0^\infty x e^{-x/y} dx\\
      &= 2y \int\limits_0^\infty e^{-x/y} dx\\
      &= 2y^2
    \end{align*}
  \end{homeworkSection}
\end{homeworkProblem}

%=============================p.376 #51==========================% 
\newpage
\begin{homeworkProblem}
  Given $X$ and $Y$ have joint density function
  \[
    f_( x, y) = \begin{array}{l l}
      \frac{ e^{-y}}{ y} & 0 < x < y < \infty\\
    \end{array}
  \]
  Find $E[ X^3|Y = y]$
  \begin{homeworkSection}{Solution}
    \begin{align*}
      E[ X^3|Y = y]
        &= \int\limits_0^\infty x^3 f(x | Y = y) dx\\
      \text{Solving for $f(x | Y = y)$}\\
      f(x | Y = y) 
        &= \frac{ f( x, y)}{ f( y)}\\
        \text{where}\\
        f( y) &= \int\limits_0^y f( x, y) dx\\
        &= \int\limits_0^y \frac{ e^{-y}}{ y} dx\\
        &= e^{-y}\\
        \text{hence}\\
      \frac{ f( x, y)}{ f( y)}
        &= \frac{ e^{y}}{ y e^{y}}\\
        &= \frac{ 1}{ y}\\
    \end{align*}
    \begin{align*}
      \text{pluggin into our expectation}\\
      \int\limits_0^\infty x^3 f(x | Y = y) dx\\
      &= \int\limits_0^\infty x^3 \frac{ 1}{ y} dx\\
      &= \frac{ y^3}{ 4}
    \end{align*}
  \end{homeworkSection}
  \end{homeworkSection}
\end{homeworkProblem}
  
\end{spacing}
\end{document}

\begin{comment}%==========================================================
%=============================Problemi==========================% 
\newpage
\begin{homeworkProblem}
  \begin{homeworkSection}{Solution}
    
  \end{homeworkSection}
\end{homeworkProblem}
%=============================Problemi==========================% 
\newpage
\begin{homeworkProblem}
  
  \begin{enumerate}[(a)]
    \item 
      \begin{homeworkSection}{Solution}
    
      \end{homeworkSection}
  \end{enumerate}
\end{homeworkProblem}

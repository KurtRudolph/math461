\documentclass{article}
\usepackage{hw_style}
\usepackage{enumerate}
\usepackage{graphicx}
\usepackage{verbatim}
\usepackage{amsmath}
\usepackage{mathtools}

% Homework Specific Information
\newcommand{\hmwkTitle}{Homework \#13}
\newcommand{\hmwkDueDate}{Monday, July 30, 2012}
\newcommand{\hmwkAuthorName}{Kurt Rudolph}%Name:
\newcommand{\hmwkNetID}{rudolph9}%your netid
\newcommand{\hmwkNotes}{}%I worked with...

\newcommand{\hmwkSubTitle}{}
\newcommand{\hmwkClass}{Math 461}
\newcommand{\hmwkClassTime}{}
\newcommand{\hmwkClassInstructor}{Kenneth B. Stolarsky}

\begin{document}
\begin{spacing}{1.1}
\maketitle

%=============================p.373 #3==========================% 
\newpage
\begin{homeworkProblem}
  {\bf Chapter 7, Problem 3}
  Gambles are independent, and each one results in
  the player being equally likely to win or lose 1 unit. 
  Let $W$ denote the net winnings of a gambler whose strategy 
  is to stop gambling immediately after his first win. Find
  \begin{enumerate}[(a)]
    \item $P\{ W > 0 \}$
      \begin{homeworkSection}{Solution}
        Given, the gambler stops immediately after his
        first win, we only need to look at one trial to determine $P\{ W > 0 \}$.
        Since $P\{ W\} = \frac{ 1}{ 2}$, therefore $P\{ W > 0\} = \frac{ 1}{ 2}$ and 
      \end{homeworkSection}
    \item $P\{ W < 0 \}$
      \begin{homeworkSection}{Solution}
        Since our gambler will gamble until he receives a win, where he will gain
        one unit, hence must receive at least two consecutive initial losses in
        oder for $W$ to be negative, therefore
        $P\{ W < 0 \} = \left(\frac{ 1}{ 2}\right)\left(\frac{ 1}{ 2}\right) = \frac{ 1}{ 4}$.
      \end{homeworkSection}
    \item $E[ W]$
      \begin{homeworkSection}{Solution}
        Let $W$ be the amount of \emph{winnings} and let $X$ be the number of trials until the
        first positive earning win is received.
        \begin{align*}
          W &= 1 - (X - 1)\\
          &= 2 - X\\
        \end{align*}
        Since the probability of receiving a positive earning win is $\frac{ 1}{ 2}$ and the
        random variable $X$ is a geometric random variable we find
        \begin{align*}
          E[ X] &= \frac{ 1}{ p}\\ 
          &= \frac{ 1}{ 1/2} \\
          &= 2
        \end{align*}
        Therefore
        \begin{align*}
          E[ W] &= E[ 2 - X]\\
          &= 2 - E[ X]\\
          &= 0
        \end{align*}
      \end{homeworkSection}
  \end{enumerate}
\end{homeworkProblem}

%=============================p.373 #4==========================% 
\newpage
\begin{homeworkProblem}
  {\bf Chapter 7, Problem 4}
  Given $X$ and $Y$ have joint density function
  \[
    f_{X, Y}( x, y) = \left\{ \begin{array}{l l}
      1/y &0 < x < y < 1\\
      0   &\text{otherwise}
    \end{array}\right.
  \]
  find
  \begin{enumerate}[(a)]
    \item $E[XY]$
      \begin{homeworkSection}{Solution}
        \begin{align*}
          E[ XY] &=  \int\limits_0^1 \int\limits_0^y xy\frac{ 1}{ y} dx dy\\
          E[ XY] &=  \int\limits_0^1 \int\limits_0^y x dx dy\\
          &= \int\limits_0^1 \frac{ y^2}{ 2} dy\\
          &= \frac{ 1}{ 6}\\
        \end{align*}
      \end{homeworkSection}
    \item $E[X]$
      \begin{homeworkSection}{Solution}
        \begin{align*}
          E[ X] &=  \int\limits_0^1 \int\limits_0^y x\frac{ 1}{ y} dx dy\\
          &= \int\limits_0^1 \frac{ y}{ 2} dy\\
          &= \frac{ 1}{ 4}\\
        \end{align*}
      \end{homeworkSection}
    \item $E[Y]$
      \begin{homeworkSection}{Solution}
        \begin{align*}
          E[ Y] &=  \int\limits_0^1 \int\limits_0^y y\frac{ 1}{ y} dx dy\\
          &=  \int\limits_0^1 \int\limits_0^y 1 dx dy\\
          &=  \int\limits_0^1 y dy\\
          &=  \frac{ 1}{ 2}\\
        \end{align*}
      \end{homeworkSection}
  \end{enumerate}
\end{homeworkProblem}

%=============================p.376 #40==========================% 
\newpage
\begin{homeworkProblem}
  Given $X$ and $Y$ have joint density function
  \[
    f_{X, Y}( x, y) = \begin{array}{l l}
      \frac{ 1}{ y} e^{-(y + \frac{ x}{ y})} & x < 0 < y\\
    \end{array}
  \]
  \begin{enumerate}[(a)]
    \item Find $E[X]$
      \begin{homeworkSection}{Solution}
      \end{homeworkSection}
    \item Find $E[Y]$
      \begin{homeworkSection}{Solution}
      \end{homeworkSection}
    \item Show $Cov( X, Y) = 1$
      \begin{homeworkSection}{Solution}
      \end{homeworkSection}
  \end{enumerate}
\end{homeworkProblem}

%=============================p.376 #48==========================% 
\newpage
\begin{homeworkProblem}
  A fair die is successively rolled. Let $X$ and $Y$ denote, 
  respectively, the number of rolls necessary to obtain a 
  6 and a 5. Find
  \begin{enumerate}[(a)]
    \item Find $E[X]$
      \begin{homeworkSection}{Solution}
      \end{homeworkSection}
    \item Find $E[X|Y = 1]$
      \begin{homeworkSection}{Solution}
      \end{homeworkSection}
    \item Show $E[X|Y = 5]$
      \begin{homeworkSection}{Solution}
      \end{homeworkSection}
  \end{enumerate}
\end{homeworkProblem}

%=============================p.376 #50==========================% 
\newpage
\begin{homeworkProblem}
  Given $X$ and $Y$ have joint density function
  \[
    f_( x, y) = \begin{array}{l l}
      \frac{ e^{-x/y} e^{-y}}{ y} & 0 < x,y < \infty\\
    \end{array}
  \]
  Find $E[ X^2|Y = y]$
  \begin{homeworkSection}{Solution}
    
  \end{homeworkSection}
\end{homeworkProblem}

%=============================p.376 #51==========================% 
\newpage
\begin{homeworkProblem}
  Given $X$ and $Y$ have joint density function
  \[
    f_( x, y) = \begin{array}{l l}
      \frac{ e^{-y}}{ y} & 0 < x < y < \infty\\
    \end{array}
  \]
  Find $E[ X^3|Y = y]$
  \begin{homeworkSection}{Solution}

  \end{homeworkSection}
\end{homeworkProblem}
  
\end{spacing}
\end{document}

\begin{comment}%==========================================================
%=============================Problemi==========================% 
\newpage
\begin{homeworkProblem}
  \begin{homeworkSection}{Solution}
    
  \end{homeworkSection}
\end{homeworkProblem}
%=============================Problemi==========================% 
\newpage
\begin{homeworkProblem}
  
  \begin{enumerate}[(a)]
    \item 
      \begin{homeworkSection}{Solution}
    
      \end{homeworkSection}
  \end{enumerate}
\end{homeworkProblem}
